\part{Scheduling}% ===> this file was generated automatically by noweave --- better not edit it
\label{part-scheduling}

\chapter{Introduction}
\label{sched-introduction}

\renewcommand{\thepage}{\arabic{page}}
\setcounter{page}{1}

Given a vehicle in progress of servicing customers that have been assigned to
it, how should the vehicle move if a new customer is added to its assignments?
One way to solve this problem is to view it as two sub-problems,
\emph{scheduling} and \emph{routing}. Scheduling is the first sub-problem. The
goal of scheduling is to determine the order that the customer pick-up and
drop-off events should be attended. Then in the routing problem, the goal is to
construct a path through the road network for the vehicle to follow so that it
will attend the events in the determined order. In this part of the examples,
we'll see some methods for solving the scheduling sub-problem. Then in the next
part, we'll see some methods for routing.

Before we begin, take note that no one knows which solutions to the scheduling
sub-problem are good for any particular situation (by using Jargo, maybe we can
find out!), thus we won't aim at satisfying an objective. Instead we'll
brainstorm a few methods and go over some that appear in the literature, and
our attention will be on \emph{how to implement these methods in Jargo}. You
will find each method described in its own chapter. At the top of each chapter,
a table will appear showing the required memory objects and side effects of the
method.

We'd be wise to define the problem \emph{in terms of Jargo data structures} so
that low-level implementation can be understood.

\nwfilename{src/sched-introduction.nw}\nwfilename{src/sched-prepend.nw}\nwbegindocs{0}\chapter{Prepend}
\label{sched-prepend}

\begin{center}
\begin{tabular}{|p{74mm}|p{74mm}|}
\hline
\textbf{Required Memory Values} & \textbf{Side Effects} \\
\hline
\begin{itemize}[leftmargin=*]
\item {\Tt{}int[]\ brem\nwendquote}: server's remaining schedule
\item {\Tt{}int[]\ bnew\nwendquote}: (empty)
\item {\Tt{}int\ rid\nwendquote}: request identifier
\item {\Tt{}int\ ro\nwendquote}: request origin vertex
\item {\Tt{}int\ rd\nwendquote}: request destination vertex
\end{itemize} &
\begin{itemize}[leftmargin=*]
\item {\Tt{}int[]\ bnew\nwendquote}: schedule after prepend
\end{itemize} \\
\hline
\end{tabular}
\end{center}

The Prepend procedure copies schedule {\Tt{}brem\nwendquote} into schedule {\Tt{}bnew\nwendquote},
\textit{but with request {\Tt{}rid\nwendquote}'s pick-up and drop-off events prepended to the
front}. An ``event'' in {\Tt{}bnew\nwendquote} is a 3-tuple, $(t,v,L)$, where $t$ is the time
of the event, $v$ is the location, and $L$ is the label describing the event.
As event times cannot be known until the route is constructed through the
schedule, {\Tt{}bnew\nwendquote} has empty time components (the 0th, 3rd, 6th, ... indices).

The procedure has three steps. We don't want to pollute the memory space, so
we'll wrap the entire procedure in braces.

\nwenddocs{}\nwbegincode{1}\sublabel{NW31whEh-3w7K97-1}\nwmargintag{{\nwtagstyle{}\subpageref{NW31whEh-3w7K97-1}}}\moddef{Scheduling: Prepend~{\nwtagstyle{}\subpageref{NW31whEh-3w7K97-1}}}\endmoddef\nwstartdeflinemarkup\nwusesondefline{\\{NW2surku-154Axj-1}\\{NW2surku-1CRkLP-1}}\nwenddeflinemarkup
\{
  \LA{}Debug: echo prepend~{\nwtagstyle{}\subpageref{NW31whEh-2pE9cy-1}}\RA{}
  \LA{}Prepend: Step 1: Initialize~{\nwtagstyle{}\subpageref{NW31whEh-4UoZD7-1}}\RA{}
  \LA{}Prepend: Step 2: Prepend~{\nwtagstyle{}\subpageref{NW31whEh-1EOgrK-1}}\RA{}
  \LA{}Prepend: Step 3: Copy~{\nwtagstyle{}\subpageref{NW31whEh-2CIafx-1}}\RA{}
\}
\nwused{\\{NW2surku-154Axj-1}\\{NW2surku-1CRkLP-1}}\nwendcode{}

\nwixlogsorted{c}{{Accept server solution}{NWkwPRG-1EyDWB-1}{\nwixd{NWkwPRG-1EyDWB-1}\nwixu{NWkwPRG-2OWvsM-1}}}%
\nwixlogsorted{c}{{Accept single solution}{NWkwPRG-47ILqi-1}{\nwixu{NWkwPRG-4dVkBS-1}\nwixu{NWkwPRG-22IadP-1}\nwixd{NWkwPRG-47ILqi-1}}}%
\nwixlogsorted{c}{{Broadway.java}{NW1QgbvE-3asg0V-1}{\nwixd{NW1QgbvE-3asg0V-1}}}%
\nwixlogsorted{c}{{Case 1: server is idle or heading towards own destination}{NWkwPRG-4dVkBS-1}{\nwixd{NWkwPRG-4dVkBS-1}\nwixu{NWkwPRG-2OWvsM-1}}}%
\nwixlogsorted{c}{{Case 2: server is heading towards a customer}{NWkwPRG-22IadP-1}{\nwixd{NWkwPRG-22IadP-1}\nwixu{NWkwPRG-2OWvsM-1}}}%
\nwixlogsorted{c}{{Check capacity}{NWkwPRG-27P27q-1}{\nwixu{NWkwPRG-3KjiZY-1}\nwixd{NWkwPRG-27P27q-1}}}%
\nwixlogsorted{c}{{Check feasible cost}{NWkwPRG-UGv0L-1}{\nwixd{NWkwPRG-UGv0L-1}\nwixu{NWkwPRG-23iwWy-1}}}%
\nwixlogsorted{c}{{Check feasible time window}{NWkwPRG-4PssH0-1}{\nwixd{NWkwPRG-4PssH0-1}\nwixu{NWkwPRG-23iwWy-1}}}%
\nwixlogsorted{c}{{Compute sequence}{NWkwPRG-3KjiZY-1}{\nwixu{NWkwPRG-22IadP-1}\nwixd{NWkwPRG-3KjiZY-1}}}%
\nwixlogsorted{c}{{Construct w from cache}{NWkwPRG-4ACdbX-1}{\nwixu{NWkwPRG-29TDYN-1}\nwixd{NWkwPRG-4ACdbX-1}}}%
\nwixlogsorted{c}{{Debug: copy}{NW31whEh-3FT9CC-1}{\nwixu{NW31whEh-2CIafx-1}\nwixd{NW31whEh-3FT9CC-1}}}%
\nwixlogsorted{c}{{Debug: copy after}{NW1UKIGb-3pgnQf-1}{\nwixu{NW1UKIGb-4QuDwc-1}\nwixd{NW1UKIGb-3pgnQf-1}}}%
\nwixlogsorted{c}{{Debug: delta filter}{NW2surku-2I20iB-1}{\nwixu{NW2surku-1OsYBb-1}\nwixd{NW2surku-2I20iB-1}}}%
\nwixlogsorted{c}{{Debug: echo NearestA case 1}{NW2surku-1KFBk4-1}{\nwixu{NW2surku-2iOiaE-1}\nwixd{NW2surku-1KFBk4-1}}}%
\nwixlogsorted{c}{{Debug: echo NearestA case 2}{NW2surku-3f0c4y-1}{\nwixu{NW2surku-2iOiaE-1}\nwixd{NW2surku-3f0c4y-1}}}%
\nwixlogsorted{c}{{Debug: echo NearestA case 3}{NW2surku-vpgYW-1}{\nwixu{NW2surku-2iOiaE-1}\nwixd{NW2surku-vpgYW-1}}}%
\nwixlogsorted{c}{{Debug: echo prepend}{NW31whEh-2pE9cy-1}{\nwixu{NW31whEh-3w7K97-1}\nwixd{NW31whEh-2pE9cy-1}}}%
\nwixlogsorted{c}{{Debug: echo prepend after}{NW1UKIGb-euTSw-1}{\nwixu{NW1UKIGb-3ujU2A-1}\nwixd{NW1UKIGb-euTSw-1}}}%
\nwixlogsorted{c}{{Debug: echo shortest path}{NW8tIqu-4ItA5T-1}{\nwixu{NW8tIqu-QuaVZ-1}\nwixd{NW8tIqu-4ItA5T-1}}}%
\nwixlogsorted{c}{{Debug: find minimum-range candidate}{NW2surku-3aXcFM-1}{\nwixu{NW2surku-i3BV8-1}\nwixd{NW2surku-3aXcFM-1}}}%
\nwixlogsorted{c}{{Debug: finish routing}{NW2surku-28JmJw-1}{\nwixu{NW2surku-dWW2r-1}\nwixd{NW2surku-28JmJw-1}}}%
\nwixlogsorted{c}{{Debug: handle request}{NW2surku-33A3UY-1}{\nwixu{NW2surku-4MdRSG-1}\nwixd{NW2surku-33A3UY-1}\nwixu{NW2rgQ87-1Jb08C-1}}}%
\nwixlogsorted{c}{{Debug: initialize}{NW31whEh-CxuR0-1}{\nwixu{NW31whEh-4UoZD7-1}\nwixd{NW31whEh-CxuR0-1}}}%
\nwixlogsorted{c}{{Debug: initialize legs container}{NW8tIqu-3R3yVH-1}{\nwixu{NW8tIqu-3lMPNM-1}\nwixd{NW8tIqu-3R3yVH-1}}}%
\nwixlogsorted{c}{{Debug: initialize NearestA}{NW2surku-m1x9e-1}{\nwixu{NW2surku-12kOmE-1}\nwixd{NW2surku-m1x9e-1}}}%
\nwixlogsorted{c}{{Debug: join route components}{NW8tIqu-J6KjD-1}{\nwixu{NW8tIqu-1OT4xQ-1}\nwixd{NW8tIqu-J6KjD-1}}}%
\nwixlogsorted{c}{{Debug: prepare scheduling and routing}{NW2surku-fxfvb-1}{\nwixu{NW2surku-lZU7g-1}\nwixd{NW2surku-fxfvb-1}}}%
\nwixlogsorted{c}{{Debug: prepend}{NW31whEh-oDZcr-1}{\nwixu{NW31whEh-1EOgrK-1}\nwixd{NW31whEh-oDZcr-1}}}%
\nwixlogsorted{c}{{Debug: prepend after}{NW1UKIGb-1muoY7-1}{\nwixu{NW1UKIGb-12A6K9-1}\nwixd{NW1UKIGb-1muoY7-1}}}%
\nwixlogsorted{c}{{Debug: put candidate}{NW2surku-1Lo26M-1}{\nwixu{NW2surku-1OsYBb-1}\nwixd{NW2surku-1Lo26M-1}}}%
\nwixlogsorted{c}{{Debug: put server location}{NW2surku-hcsfv-1}{\nwixu{NW2surku-1xARKZ-1}\nwixd{NW2surku-hcsfv-1}}}%
\nwixlogsorted{c}{{Debug: range filter}{NW2surku-4aukgD-1}{\nwixu{NW2surku-1OsYBb-1}\nwixd{NW2surku-4aukgD-1}}}%
\nwixlogsorted{c}{{Debug: reset}{NW2surku-1iDlmC-1}{\nwixu{NW2surku-3xagbK-1}\nwixd{NW2surku-1iDlmC-1}}}%
\nwixlogsorted{c}{{Debug: set first leg}{NW8tIqu-1F8aiy-1}{\nwixu{NW8tIqu-3lMPNM-1}\nwixd{NW8tIqu-1F8aiy-1}}}%
\nwixlogsorted{c}{{Debug: set initial waypoint}{NW2surku-1GjT93-1}{\nwixu{NW2surku-dWW2r-1}\nwixd{NW2surku-1GjT93-1}}}%
\nwixlogsorted{c}{{Debug: set leg}{NW8tIqu-2dVika-1}{\nwixu{NW8tIqu-2EZ2CY-1}\nwixd{NW8tIqu-2dVika-1}}}%
\nwixlogsorted{c}{{Debug: set ok}{NW2surku-2embq0-1}{\nwixu{NW2surku-154Axj-1}\nwixu{NW2surku-3J1EB5-1}\nwixu{NW2surku-1CRkLP-1}\nwixd{NW2surku-2embq0-1}}}%
\nwixlogsorted{c}{{Debug: set route length and time}{NW8tIqu-3WOc2v-1}{\nwixu{NW8tIqu-3lMPNM-1}\nwixu{NW8tIqu-2EZ2CY-1}\nwixd{NW8tIqu-3WOc2v-1}}}%
\nwixlogsorted{c}{{Debug: set schedule time}{NW8tIqu-1EJRY4-1}{\nwixu{NW8tIqu-1nl6rv-1}\nwixd{NW8tIqu-1EJRY4-1}}}%
\nwixlogsorted{c}{{Debug: set vehicle end time}{NW8tIqu-3ZHBo5-1}{\nwixu{NW8tIqu-1nl6rv-1}\nwixd{NW8tIqu-3ZHBo5-1}}}%
\nwixlogsorted{c}{{Debug: submit assignment}{NW2surku-2uYsdK-1}{\nwixu{NW2surku-iFIwh-1}\nwixd{NW2surku-2uYsdK-1}}}%
\nwixlogsorted{c}{{GreedyInsertion.java}{NWkwPRG-1Tuf6f-1}{\nwixd{NWkwPRG-1Tuf6f-1}}}%
\nwixlogsorted{c}{{GreedyInsertion.java preamble}{NWkwPRG-1neuor-1}{\nwixu{NWkwPRG-1Tuf6f-1}\nwixd{NWkwPRG-1neuor-1}}}%
\nwixlogsorted{c}{{GreedyInsertion: computeCost(6)}{NWkwPRG-23iwWy-1}{\nwixu{NWkwPRG-1Tuf6f-1}\nwixd{NWkwPRG-23iwWy-1}}}%
\nwixlogsorted{c}{{GreedyInsertion: end(0)}{NWkwPRG-222iyz-1}{\nwixu{NWkwPRG-1Tuf6f-1}\nwixd{NWkwPRG-222iyz-1}}}%
\nwixlogsorted{c}{{GreedyInsertion: endCollectServerLocations(1)}{NWkwPRG-1cMqeO-1}{\nwixu{NWkwPRG-1Tuf6f-1}\nwixd{NWkwPRG-1cMqeO-1}}}%
\nwixlogsorted{c}{{GreedyInsertion: handleRequest(1)}{NWkwPRG-2OWvsM-1}{\nwixu{NWkwPRG-1Tuf6f-1}\nwixd{NWkwPRG-2OWvsM-1}}}%
\nwixlogsorted{c}{{GreedyInsertion: member variables}{NWkwPRG-20xkUX-1}{\nwixu{NWkwPRG-1Tuf6f-1}\nwixd{NWkwPRG-20xkUX-1}}}%
\nwixlogsorted{c}{{Initialize global vars}{NWkwPRG-jOToH-1}{\nwixd{NWkwPRG-jOToH-1}\nwixu{NWkwPRG-2OWvsM-1}}}%
\nwixlogsorted{c}{{Initialize server vars}{NWkwPRG-3y7kuU-1}{\nwixd{NWkwPRG-3y7kuU-1}\nwixu{NWkwPRG-2OWvsM-1}}}%
\nwixlogsorted{c}{{NearestA.java}{NW2surku-zKRYu-1}{\nwixd{NW2surku-zKRYu-1}}}%
\nwixlogsorted{c}{{NearestA: Go to Reduce?}{NW2surku-MbHJM-1}{\nwixu{NW2surku-4MdRSG-1}\nwixd{NW2surku-MbHJM-1}}}%
\nwixlogsorted{c}{{NearestA: Go to Submit?}{NW2surku-35BBt9-1}{\nwixu{NW2surku-4MdRSG-1}\nwixd{NW2surku-35BBt9-1}}}%
\nwixlogsorted{c}{{NearestA: handleRequest(1)}{NW2surku-4MdRSG-1}{\nwixu{NW2surku-zKRYu-1}\nwixd{NW2surku-4MdRSG-1}}}%
\nwixlogsorted{c}{{NearestA: handleServerLocation(1)}{NW2surku-1xARKZ-1}{\nwixu{NW2surku-zKRYu-1}\nwixd{NW2surku-1xARKZ-1}\nwixu{NW2rgQ87-105L3q-1}}}%
\nwixlogsorted{c}{{NearestA: member variables}{NW2surku-2qcVD-1}{\nwixu{NW2surku-zKRYu-1}\nwixd{NW2surku-2qcVD-1}\nwixu{NW2rgQ87-105L3q-1}}}%
\nwixlogsorted{c}{{NearestA: preamble}{NW2surku-3wQFt9-1}{\nwixu{NW2surku-zKRYu-1}\nwixd{NW2surku-3wQFt9-1}\nwixu{NW2rgQ87-105L3q-1}}}%
\nwixlogsorted{c}{{NearestA: Step 1: Initialize}{NW2surku-12kOmE-1}{\nwixu{NW2surku-4MdRSG-1}\nwixd{NW2surku-12kOmE-1}\nwixu{NW2rgQ87-1Jb08C-1}}}%
\nwixlogsorted{c}{{NearestA: Step 2: Map and Filter}{NW2surku-1OsYBb-1}{\nwixu{NW2surku-4MdRSG-1}\nwixd{NW2surku-1OsYBb-1}\nwixu{NW2rgQ87-1Jb08C-1}}}%
\nwixlogsorted{c}{{NearestA: Step 3: Reduce}{NW2surku-i3BV8-1}{\nwixu{NW2surku-4MdRSG-1}\nwixd{NW2surku-i3BV8-1}\nwixu{NW2rgQ87-1Jb08C-1}}}%
\nwixlogsorted{c}{{NearestA: Step 4: Prepare}{NW2surku-lZU7g-1}{\nwixu{NW2surku-4MdRSG-1}\nwixd{NW2surku-lZU7g-1}}}%
\nwixlogsorted{c}{{NearestA: Step 5: Case 1}{NW2surku-154Axj-1}{\nwixu{NW2surku-2iOiaE-1}\nwixd{NW2surku-154Axj-1}}}%
\nwixlogsorted{c}{{NearestA: Step 5: Case 2}{NW2surku-3J1EB5-1}{\nwixu{NW2surku-2iOiaE-1}\nwixd{NW2surku-3J1EB5-1}}}%
\nwixlogsorted{c}{{NearestA: Step 5: Case 3}{NW2surku-1CRkLP-1}{\nwixu{NW2surku-2iOiaE-1}\nwixd{NW2surku-1CRkLP-1}}}%
\nwixlogsorted{c}{{NearestA: Step 5: Prepare Routing}{nw@notdef}{\nwixu{NW2rgQ87-My4EY-1}}}%
\nwixlogsorted{c}{{NearestA: Step 5: Schedule}{NW2surku-2iOiaE-1}{\nwixu{NW2surku-4MdRSG-1}\nwixd{NW2surku-2iOiaE-1}}}%
\nwixlogsorted{c}{{NearestA: Step 6: Route}{NW2surku-dWW2r-1}{\nwixu{NW2surku-4MdRSG-1}\nwixd{NW2surku-dWW2r-1}}}%
\nwixlogsorted{c}{{NearestA: Step 7: Submit}{NW2surku-iFIwh-1}{\nwixu{NW2surku-4MdRSG-1}\nwixd{NW2surku-iFIwh-1}}}%
\nwixlogsorted{c}{{NearestA: Step 8: Reset}{NW2surku-3xagbK-1}{\nwixu{NW2surku-4MdRSG-1}\nwixd{NW2surku-3xagbK-1}}}%
\nwixlogsorted{c}{{NearestB.java}{NW2rgQ87-105L3q-1}{\nwixd{NW2rgQ87-105L3q-1}}}%
\nwixlogsorted{c}{{NearestB: handleRequest(1)}{NW2rgQ87-1Jb08C-1}{\nwixu{NW2rgQ87-105L3q-1}\nwixd{NW2rgQ87-1Jb08C-1}}}%
\nwixlogsorted{c}{{NearestB: Step 5: Prepare Routing}{NW2rgQ87-My4EY-1}{\nwixd{NW2rgQ87-My4EY-1}}}%
\nwixlogsorted{c}{{Prepend: Step 1: Initialize}{NW31whEh-4UoZD7-1}{\nwixu{NW31whEh-3w7K97-1}\nwixd{NW31whEh-4UoZD7-1}\nwixu{NW1UKIGb-3ujU2A-1}}}%
\nwixlogsorted{c}{{Prepend: Step 2: Prepend}{NW31whEh-1EOgrK-1}{\nwixu{NW31whEh-3w7K97-1}\nwixd{NW31whEh-1EOgrK-1}}}%
\nwixlogsorted{c}{{Prepend: Step 2: Prepend After}{NW1UKIGb-12A6K9-1}{\nwixu{NW1UKIGb-3ujU2A-1}\nwixd{NW1UKIGb-12A6K9-1}}}%
\nwixlogsorted{c}{{Prepend: Step 3: Copy}{NW31whEh-2CIafx-1}{\nwixu{NW31whEh-3w7K97-1}\nwixd{NW31whEh-2CIafx-1}}}%
\nwixlogsorted{c}{{Prepend: Step 3: Copy After}{NW1UKIGb-4QuDwc-1}{\nwixu{NW1UKIGb-3ujU2A-1}\nwixd{NW1UKIGb-4QuDwc-1}}}%
\nwixlogsorted{c}{{Routing: Shortest Path}{NW8tIqu-QuaVZ-1}{\nwixd{NW8tIqu-QuaVZ-1}\nwixu{NW2surku-dWW2r-1}}}%
\nwixlogsorted{c}{{Scheduling: Prepend}{NW31whEh-3w7K97-1}{\nwixd{NW31whEh-3w7K97-1}\nwixu{NW2surku-154Axj-1}\nwixu{NW2surku-1CRkLP-1}}}%
\nwixlogsorted{c}{{Scheduling: Prepend After}{NW1UKIGb-3ujU2A-1}{\nwixd{NW1UKIGb-3ujU2A-1}\nwixu{NW2surku-3J1EB5-1}}}%
\nwixlogsorted{c}{{Shortest: Step 1: Initialize}{NW8tIqu-3lMPNM-1}{\nwixu{NW8tIqu-QuaVZ-1}\nwixd{NW8tIqu-3lMPNM-1}}}%
\nwixlogsorted{c}{{Shortest: Step 2: Compute}{NW8tIqu-2EZ2CY-1}{\nwixu{NW8tIqu-QuaVZ-1}\nwixd{NW8tIqu-2EZ2CY-1}}}%
\nwixlogsorted{c}{{Shortest: Step 3: Join}{NW8tIqu-1OT4xQ-1}{\nwixu{NW8tIqu-QuaVZ-1}\nwixd{NW8tIqu-1OT4xQ-1}}}%
\nwixlogsorted{c}{{Shortest: Step 4: Fill}{NW8tIqu-1nl6rv-1}{\nwixu{NW8tIqu-QuaVZ-1}\nwixd{NW8tIqu-1nl6rv-1}}}%
\nwixlogsorted{c}{{Submit global solution}{NWkwPRG-29TDYN-1}{\nwixd{NWkwPRG-29TDYN-1}\nwixu{NWkwPRG-2OWvsM-1}}}%
\nwbegindocs{2}\nwdocspar

\section{Step 1: Initialize}

First we'll initialize {\Tt{}bnew\nwendquote} to the correct size. The correct size is
calculated as follows. In Jargo, a server's remaining schedule is returned as a
4-tuple flattened array of $(t,v,Ls,Lr)$ components. Each 4-tuple is one event.
To get the number of events, we divide the length of the remaining schedule
{\Tt{}brem\nwendquote} by four. Then we add two to this number because the new schedule will
have two additional events, one pick-up and one drop-off. Finally we multiply
this sum by three. (TODO: change remaining-schedule and submitted-schedule to
be in the same format!)

\nwenddocs{}\nwbegincode{3}\sublabel{NW31whEh-4UoZD7-1}\nwmargintag{{\nwtagstyle{}\subpageref{NW31whEh-4UoZD7-1}}}\moddef{Prepend: Step 1: Initialize~{\nwtagstyle{}\subpageref{NW31whEh-4UoZD7-1}}}\endmoddef\nwstartdeflinemarkup\nwusesondefline{\\{NW31whEh-3w7K97-1}\\{NW1UKIGb-3ujU2A-1}}\nwenddeflinemarkup
final int m = (brem.length/4);
bnew = new int[(3*(m + 2))];
\LA{}Debug: initialize~{\nwtagstyle{}\subpageref{NW31whEh-CxuR0-1}}\RA{}
\nwused{\\{NW31whEh-3w7K97-1}\\{NW1UKIGb-3ujU2A-1}}\nwendcode{}\nwbegindocs{4}\nwdocspar

\section{Step 2: Prepend}

Now we'll do the prepend by populating the first six elements in {\Tt{}bnew\nwendquote}.
Elements {\Tt{}bnew[0..2]\nwendquote} represent the request pick-up event, and elements
{\Tt{}bnew[3..5]\nwendquote} represent the request drop-off event. As event times are
unknown, we'll leave out the 0th and 3rd indices.

\nwenddocs{}\nwbegincode{5}\sublabel{NW31whEh-1EOgrK-1}\nwmargintag{{\nwtagstyle{}\subpageref{NW31whEh-1EOgrK-1}}}\moddef{Prepend: Step 2: Prepend~{\nwtagstyle{}\subpageref{NW31whEh-1EOgrK-1}}}\endmoddef\nwstartdeflinemarkup\nwusesondefline{\\{NW31whEh-3w7K97-1}}\nwenddeflinemarkup
bnew[1] = ro;
bnew[2] = rid;
bnew[4] = rd;
bnew[5] = rid;
\LA{}Debug: prepend~{\nwtagstyle{}\subpageref{NW31whEh-oDZcr-1}}\RA{}
\nwused{\\{NW31whEh-3w7K97-1}}\nwendcode{}\nwbegindocs{6}\nwdocspar

\section{Step 3: Copy}

Just like that, we're almost done. In the third step, we'll simply copy over
{\Tt{}brem\nwendquote} into {\Tt{}bnew\nwendquote} while shifting the destination indice by six. We shift
by six to account for the two 3-tuples we inserted in Step 2. (We have to
``squeeze'' the label in {\Tt{}brem\nwendquote} into {\Tt{}bnew\nwendquote}. The label in {\Tt{}brem\nwendquote} takes
two elements while it only takes one element in {\Tt{}bnew\nwendquote}. I'll fix this
inconsistency in a later version!)

\nwenddocs{}\nwbegincode{7}\sublabel{NW31whEh-2CIafx-1}\nwmargintag{{\nwtagstyle{}\subpageref{NW31whEh-2CIafx-1}}}\moddef{Prepend: Step 3: Copy~{\nwtagstyle{}\subpageref{NW31whEh-2CIafx-1}}}\endmoddef\nwstartdeflinemarkup\nwusesondefline{\\{NW31whEh-3w7K97-1}}\nwenddeflinemarkup
for (int i = 0; i < m; i++) \{
  // Extract vertex and labels
  final int bv = brem[(4*i + 1)];
  final int ls = brem[(4*i + 2)];
  final int lr = brem[(4*i + 3)];
  // Copy into bnew
  bnew[(3*i + 7)] = bv;
  bnew[(3*i + 8)] = (ls == 0 ? lr : ls);  // <-- squeeze the label
  \LA{}Debug: copy~{\nwtagstyle{}\subpageref{NW31whEh-3FT9CC-1}}\RA{}
\}
\nwused{\\{NW31whEh-3w7K97-1}}\nwendcode{}\nwbegindocs{8}\nwdocspar

\section{Example}

Here is an example of the procedure in action. (TODO)

\section{Full Code}

Here is the full code.

\begin{verbatim}
{
  final int m = (brem.length/4);
  bnew = new int[(3*(m + 2))];
  bnew[1] = ro;
  bnew[2] = rid;
  bnew[4] = rd;
  bnew[5] = rid;
  for (int i = 0; i < m; i++) {
    // Extract vertex and labels
    final int bv = brem[(4*i + 1)];
    final int ls = brem[(4*i + 2)];
    final int lr = brem[(4*i + 3)];
    // Copy into bnew
    bnew[(3*i + 7)] = bv;
    bnew[(3*i + 8)] = (ls == 0 ? lr : ls);  // <-- squeeze the label
  }
}
\end{verbatim}

\section{Debug}

Printing debugging messages to screen can be useful for tracking execution of
the procedure. The below print statements can be used to inspect {\Tt{}bnew\nwendquote} as
its contents change.

\subsection{Debug: Echo Prepend}

\nwenddocs{}\nwbegincode{9}\sublabel{NW31whEh-2pE9cy-1}\nwmargintag{{\nwtagstyle{}\subpageref{NW31whEh-2pE9cy-1}}}\moddef{Debug: echo prepend~{\nwtagstyle{}\subpageref{NW31whEh-2pE9cy-1}}}\endmoddef\nwstartdeflinemarkup\nwusesondefline{\\{NW31whEh-3w7K97-1}}\nwenddeflinemarkup
if (DEBUG) \{
  System.out.printf("prepend\\n");
\}
\nwused{\\{NW31whEh-3w7K97-1}}\nwendcode{}\nwbegindocs{10}\nwdocspar

\subsection{Debug: Initialize}

\nwenddocs{}\nwbegincode{11}\sublabel{NW31whEh-CxuR0-1}\nwmargintag{{\nwtagstyle{}\subpageref{NW31whEh-CxuR0-1}}}\moddef{Debug: initialize~{\nwtagstyle{}\subpageref{NW31whEh-CxuR0-1}}}\endmoddef\nwstartdeflinemarkup\nwusesondefline{\\{NW31whEh-4UoZD7-1}}\nwenddeflinemarkup
if (DEBUG) \{
  System.out.printf("init bnew=\{ \}, length=%d\\n", bnew.length);
\}
\nwused{\\{NW31whEh-4UoZD7-1}}\nwendcode{}\nwbegindocs{12}\nwdocspar

\subsection{Debug: Prepend}

\nwenddocs{}\nwbegincode{13}\sublabel{NW31whEh-oDZcr-1}\nwmargintag{{\nwtagstyle{}\subpageref{NW31whEh-oDZcr-1}}}\moddef{Debug: prepend~{\nwtagstyle{}\subpageref{NW31whEh-oDZcr-1}}}\endmoddef\nwstartdeflinemarkup\nwusesondefline{\\{NW31whEh-1EOgrK-1}}\nwenddeflinemarkup
if (DEBUG) \{
  System.out.printf("set bnew[1]=%d\\n", bnew[1]);
  System.out.printf("set bnew[2]=%d\\n", bnew[2]);
  System.out.printf("set bnew[4]=%d\\n", bnew[4]);
  System.out.printf("set bnew[5]=%d\\n", bnew[5]);
\}
\nwused{\\{NW31whEh-1EOgrK-1}}\nwendcode{}\nwbegindocs{14}\nwdocspar

\subsection{Debug: Copy}

\nwenddocs{}\nwbegincode{15}\sublabel{NW31whEh-3FT9CC-1}\nwmargintag{{\nwtagstyle{}\subpageref{NW31whEh-3FT9CC-1}}}\moddef{Debug: copy~{\nwtagstyle{}\subpageref{NW31whEh-3FT9CC-1}}}\endmoddef\nwstartdeflinemarkup\nwusesondefline{\\{NW31whEh-2CIafx-1}}\nwenddeflinemarkup
if (DEBUG) \{
  System.out.printf("set bnew[%d]=%d\\n", (3*i + 7), bnew[(3*i + 7)]);
  System.out.printf("set bnew[%d]=%d\\n", (3*i + 8), bnew[(3*i + 8)]);
\}
\nwused{\\{NW31whEh-2CIafx-1}}\nwendcode{}\nwbegindocs{16}\nwdocspar

\nwenddocs{}\nwfilename{src/sched-prepend-after.nw}\nwbegindocs{0}\chapter{Prepend After}
\label{sched-prepend-after}

The second scheduling procedure is a variation of Prepend called Prepend After.
In this variation, the pick-up and drop-off are prepended \textit{after the
first event} in the remaining schedule. The idea is to let the vehicle complete
its next event before servicing the new events. The memory table for
Prepend After is the same as for Prepend.

\begin{center}
\begin{tabular}{|p{74mm}|p{74mm}|}
\hline
\textbf{Required Memory Values} & \textbf{Side Effects} \\
\hline
\begin{itemize}[leftmargin=*]
\item {\Tt{}int[]\ brem\nwendquote}: server's remaining schedule
\item {\Tt{}int[]\ bnew\nwendquote}: (empty)
\item {\Tt{}int\ rid\nwendquote}: request identifier
\item {\Tt{}int\ ro\nwendquote}: request origin vertex
\item {\Tt{}int\ rd\nwendquote}: request destination vertex
\end{itemize} &
\begin{itemize}[leftmargin=*]
\item {\Tt{}int[]\ bnew\nwendquote}: schedule after prepend
\end{itemize} \\
\hline
\end{tabular}
\end{center}

The first step, initialization, is the same here as in Prepend.

\nwenddocs{}\nwbegincode{1}\sublabel{NW1UKIGb-3ujU2A-1}\nwmargintag{{\nwtagstyle{}\subpageref{NW1UKIGb-3ujU2A-1}}}\moddef{Scheduling: Prepend After~{\nwtagstyle{}\subpageref{NW1UKIGb-3ujU2A-1}}}\endmoddef\nwstartdeflinemarkup\nwusesondefline{\\{NW2surku-3J1EB5-1}}\nwenddeflinemarkup
\{
  \LA{}Debug: echo prepend after~{\nwtagstyle{}\subpageref{NW1UKIGb-euTSw-1}}\RA{}
  \LA{}Prepend: Step 1: Initialize~{\nwtagstyle{}\subpageref{NW31whEh-4UoZD7-1}}\RA{}
  \LA{}Prepend: Step 2: Prepend After~{\nwtagstyle{}\subpageref{NW1UKIGb-12A6K9-1}}\RA{}
  \LA{}Prepend: Step 3: Copy After~{\nwtagstyle{}\subpageref{NW1UKIGb-4QuDwc-1}}\RA{}
\}
\nwused{\\{NW2surku-3J1EB5-1}}\nwendcode{}\nwbegindocs{2}\nwdocspar

\section{Step 2: Prepend After}

In the second step, the request pick-up and drop-off events are placed in
elements {\Tt{}bnew[3..5]\nwendquote} and {\Tt{}bnew[6..8]\nwendquote}, respectively. Again, as event
times are not known until routing, we'll leave out the time components
at the 3rd and 6th indices.

\nwenddocs{}\nwbegincode{3}\sublabel{NW1UKIGb-12A6K9-1}\nwmargintag{{\nwtagstyle{}\subpageref{NW1UKIGb-12A6K9-1}}}\moddef{Prepend: Step 2: Prepend After~{\nwtagstyle{}\subpageref{NW1UKIGb-12A6K9-1}}}\endmoddef\nwstartdeflinemarkup\nwusesondefline{\\{NW1UKIGb-3ujU2A-1}}\nwenddeflinemarkup
bnew[4] = ro;
bnew[5] = rid;
bnew[7] = rd;
bnew[8] = rid;
\LA{}Debug: prepend after~{\nwtagstyle{}\subpageref{NW1UKIGb-1muoY7-1}}\RA{}
\nwused{\\{NW1UKIGb-3ujU2A-1}}\nwendcode{}\nwbegindocs{4}\nwdocspar

\section{Step 3: Copy After}

In the third step, we'll once again copy events in {\Tt{}brem\nwendquote} into {\Tt{}bnew\nwendquote}.
This time, the first event is copied without any positional shift, while later
events are shifted again by six positions.

\nwenddocs{}\nwbegincode{5}\sublabel{NW1UKIGb-4QuDwc-1}\nwmargintag{{\nwtagstyle{}\subpageref{NW1UKIGb-4QuDwc-1}}}\moddef{Prepend: Step 3: Copy After~{\nwtagstyle{}\subpageref{NW1UKIGb-4QuDwc-1}}}\endmoddef\nwstartdeflinemarkup\nwusesondefline{\\{NW1UKIGb-3ujU2A-1}}\nwenddeflinemarkup
for (int i = 0; i < m; i++) \{
  // Extract vertex and labels
  final int bv = brem[(4*i + 1)];
  final int ls = brem[(4*i + 2)];
  final int lr = brem[(4*i + 3)];
  // Copy into bnew and shift if not first event
  bnew[(3*i + (i < 1 ? 1 : 7))] = bv;
  bnew[(3*i + (i < 1 ? 2 : 8))] = (ls == 0 ? lr : ls);
  \LA{}Debug: copy after~{\nwtagstyle{}\subpageref{NW1UKIGb-3pgnQf-1}}\RA{}
\}
\nwused{\\{NW1UKIGb-3ujU2A-1}}\nwendcode{}\nwbegindocs{6}\nwdocspar

\section{Example}

Here is an example of the procedure in action. (TODO)

\section{Full Code}

Here is the full code.

\begin{verbatim}
{
  final int m = (brem.length/4);
  bnew = new int[(3*(m + 2))];
  bnew[4] = ro;
  bnew[5] = rid;
  bnew[7] = rd;
  bnew[8] = rid;
  for (int i = 0; i < m; i++) {
    // Extract vertex and labels
    final int bv = brem[(4*i + 1)];
    final int ls = brem[(4*i + 2)];
    final int lr = brem[(4*i + 3)];
    // Copy into bnew and shift if not first event
    bnew[(3*i + (i < 1 ? 1 : 7))] = bv;
    bnew[(3*i + (i < 1 ? 2 : 8))] = (ls == 0 ? lr : ls);
  }
}
\end{verbatim}

\section{Debug}

The below print statements can be used to inspect {\Tt{}bnew\nwendquote} as its contents
change.

\subsection{Debug: Echo Prepend After}

\nwenddocs{}\nwbegincode{7}\sublabel{NW1UKIGb-euTSw-1}\nwmargintag{{\nwtagstyle{}\subpageref{NW1UKIGb-euTSw-1}}}\moddef{Debug: echo prepend after~{\nwtagstyle{}\subpageref{NW1UKIGb-euTSw-1}}}\endmoddef\nwstartdeflinemarkup\nwusesondefline{\\{NW1UKIGb-3ujU2A-1}}\nwenddeflinemarkup
if (DEBUG) \{
  System.out.printf("prepend after\\n");
\}
\nwused{\\{NW1UKIGb-3ujU2A-1}}\nwendcode{}\nwbegindocs{8}\nwdocspar

\subsection{Debug: Prepend After}

\nwenddocs{}\nwbegincode{9}\sublabel{NW1UKIGb-1muoY7-1}\nwmargintag{{\nwtagstyle{}\subpageref{NW1UKIGb-1muoY7-1}}}\moddef{Debug: prepend after~{\nwtagstyle{}\subpageref{NW1UKIGb-1muoY7-1}}}\endmoddef\nwstartdeflinemarkup\nwusesondefline{\\{NW1UKIGb-12A6K9-1}}\nwenddeflinemarkup
if (DEBUG) \{
  System.out.printf("set bnew[4]=%d\\n", bnew[4]);
  System.out.printf("set bnew[5]=%d\\n", bnew[5]);
  System.out.printf("set bnew[7]=%d\\n", bnew[7]);
  System.out.printf("set bnew[8]=%d\\n", bnew[8]);
\}
\nwused{\\{NW1UKIGb-12A6K9-1}}\nwendcode{}\nwbegindocs{10}\nwdocspar

\subsection{Debug: Copy After}

\nwenddocs{}\nwbegincode{11}\sublabel{NW1UKIGb-3pgnQf-1}\nwmargintag{{\nwtagstyle{}\subpageref{NW1UKIGb-3pgnQf-1}}}\moddef{Debug: copy after~{\nwtagstyle{}\subpageref{NW1UKIGb-3pgnQf-1}}}\endmoddef\nwstartdeflinemarkup\nwusesondefline{\\{NW1UKIGb-4QuDwc-1}}\nwenddeflinemarkup
if (DEBUG) \{
  if (i < 1) \{
    System.out.printf("set bnew[%d]=%d\\n", (3*i + 1), bnew[(3*i + 1)]);
    System.out.printf("set bnew[%d]=%d\\n", (3*i + 2), bnew[(3*i + 2)]);
  \} else \{
    System.out.printf("set bnew[%d]=%d\\n", (3*i + 7), bnew[(3*i + 7)]);
    System.out.printf("set bnew[%d]=%d\\n", (3*i + 8), bnew[(3*i + 8)]);
  \}
\}
\nwused{\\{NW1UKIGb-4QuDwc-1}}\nwendcode{}\nwbegindocs{12}\nwdocspar

\nwenddocs{}\nwfilename{src/route-shortest.nw}\nwbegindocs{0}\part{Routing}
\label{part-routing}

\chapter{Shortest Path}
\label{route-shortest}

In the previous chapters, we saw some scheduling procedures for ridesharing
algorithms. Now we will take a look at routing procedures. We'll start with
Shortest Path. This procedure calculates \textit{consecutive shortest paths
between events in a schedule in sequence}. We will use the {\Tt{}Tools\nwendquote} class to
quickly calculate shortest paths using a G-tree index of the road network
graph.

We will need some other memory objects. We will need the schedule to route
through, {\Tt{}bnew\nwendquote}, produced using some scheduling procedure such as Prepend.
We will also need a container to store the computed route, {\Tt{}wnew\nwendquote}.
Finally, we need an \textit{initial waypoint} {\Tt{}wbeg\nwendquote} to start off the route.
Recall that scheduling procedures do not fill in event times because these
times depend on the routing. After we find the route here, we'll be able to
update {\Tt{}bnew\nwendquote} by filling in the event times.

\begin{center}
\begin{tabular}{|p{74mm}|p{74mm}|}
\hline
\textbf{Required Memory Values} & \textbf{Side Effects} \\
\hline
\begin{itemize}[leftmargin=*]
\item {\Tt{}Tools\ this.tools\nwendquote}: Jargo utility functions
\item {\Tt{}int[]\ bnew\nwendquote}: schedule to route through, with empty time components
\item {\Tt{}int[]\ wbeg\nwendquote}: first $(t,v)$ waypoint in the new route
\item {\Tt{}int[]\ wnew\nwendquote}: (empty)
\end{itemize} &
\begin{itemize}[leftmargin=*]
\item {\Tt{}int[]\ bnew\nwendquote}: time components added
\item {\Tt{}int[]\ wnew\nwendquote}: the computed route
\end{itemize} \\
\hline
\end{tabular}
\end{center}

The procedure takes four steps. In the first step, we'll initialize some memory
structures. In the second step, we'll compute shortest paths between adjacent
events. In the third step, we'll join the paths into a single route. In the
last step, we'll fill in the event times into {\Tt{}bnew\nwendquote}.

\nwenddocs{}\nwbegincode{1}\sublabel{NW8tIqu-QuaVZ-1}\nwmargintag{{\nwtagstyle{}\subpageref{NW8tIqu-QuaVZ-1}}}\moddef{Routing: Shortest Path~{\nwtagstyle{}\subpageref{NW8tIqu-QuaVZ-1}}}\endmoddef\nwstartdeflinemarkup\nwusesondefline{\\{NW2surku-dWW2r-1}}\nwenddeflinemarkup
\{
  \LA{}Debug: echo shortest path~{\nwtagstyle{}\subpageref{NW8tIqu-4ItA5T-1}}\RA{}
  \LA{}Shortest: Step 1: Initialize~{\nwtagstyle{}\subpageref{NW8tIqu-3lMPNM-1}}\RA{}
  \LA{}Shortest: Step 2: Compute~{\nwtagstyle{}\subpageref{NW8tIqu-2EZ2CY-1}}\RA{}
  \LA{}Shortest: Step 3: Join~{\nwtagstyle{}\subpageref{NW8tIqu-1OT4xQ-1}}\RA{}
  \LA{}Shortest: Step 4: Fill~{\nwtagstyle{}\subpageref{NW8tIqu-1nl6rv-1}}\RA{}
\}
\nwused{\\{NW2surku-dWW2r-1}}\nwendcode{}\nwbegindocs{2}\nwdocspar

\section{Step 1: Initialize}

Let's initialize the memory.  We'll need some place to store the individual
paths as they are computed, call it {\Tt{}legs\nwendquote}. A path by itself is an array of
integer vertices, so {\Tt{}legs\nwendquote} will be an array of integer arrays. For each
event, there is a path traveling to the event. The first path travels from
{\Tt{}wbeg\nwendquote} to the first event, the second path travels from the first event to
the second event, and so on. The number of events in {\Tt{}bnew\nwendquote} is its length
divided by three. The {\Tt{}legs\nwendquote} array will need to hold this many paths. We'll
store the number into {\Tt{}p\nwendquote} and use {\Tt{}p\nwendquote} to initialize {\Tt{}legs\nwendquote}.

The first path is slightly different then the rest because it starts from
{\Tt{}wbeg\nwendquote}. We'll go ahead and compute the first path now. We'll use the
{\Tt{}computeRoute\nwendquote} method on the {\Tt{}Tools\nwendquote} class to find the shortest path.  The
arguments in order are: starting vertex, ending vertex, and \textit{initial
time}. The time is needed because {\Tt{}computeRoute\nwendquote} returns a sequence of
waypoints, with time components next to each vertex. The duration between each
adjacent vertex pair is the shortest duration needed to travel between the two
vertices. This duration is computed using the maximum free-flow speed
information in the road network. (TODO! Right now the speed is hard-coded to be
10 m/s for all edges.)

After we get the first path, we'll remember its length {\Tt{}n\nwendquote} and ending time
{\Tt{}t\nwendquote} and then store it into {\Tt{}legs\nwendquote}. We will need {\Tt{}n\nwendquote} and {\Tt{}t\nwendquote} later.

\nwenddocs{}\nwbegincode{3}\sublabel{NW8tIqu-3lMPNM-1}\nwmargintag{{\nwtagstyle{}\subpageref{NW8tIqu-3lMPNM-1}}}\moddef{Shortest: Step 1: Initialize~{\nwtagstyle{}\subpageref{NW8tIqu-3lMPNM-1}}}\endmoddef\nwstartdeflinemarkup\nwusesondefline{\\{NW8tIqu-QuaVZ-1}}\nwenddeflinemarkup
final int p = (bnew.length/3);
final int[][] legs = new int[p][];
\LA{}Debug: initialize legs container~{\nwtagstyle{}\subpageref{NW8tIqu-3R3yVH-1}}\RA{}

int[] leg = this.tools.computeRoute(wbeg[1], bnew[1], wbeg[0]);
int n = leg.length;
int t = leg[(n - 2)];
\LA{}Debug: set route length and time~{\nwtagstyle{}\subpageref{NW8tIqu-3WOc2v-1}}\RA{}

legs[0] = leg;
\LA{}Debug: set first leg~{\nwtagstyle{}\subpageref{NW8tIqu-1F8aiy-1}}\RA{}
\nwused{\\{NW8tIqu-QuaVZ-1}}\nwendcode{}\nwbegindocs{4}\nwdocspar

\section{Step 2: Compute}

Now we can compute each of the paths. For each event in {\Tt{}bnew\nwendquote}, we'll compute
the path to the event traveling from the previous event. The starting time of
each path is the ending time of the previous path. We'll store each path into
{\Tt{}legs\nwendquote} and update {\Tt{}n\nwendquote} and {\Tt{}t\nwendquote} as we go along.

\nwenddocs{}\nwbegincode{5}\sublabel{NW8tIqu-2EZ2CY-1}\nwmargintag{{\nwtagstyle{}\subpageref{NW8tIqu-2EZ2CY-1}}}\moddef{Shortest: Step 2: Compute~{\nwtagstyle{}\subpageref{NW8tIqu-2EZ2CY-1}}}\endmoddef\nwstartdeflinemarkup\nwusesondefline{\\{NW8tIqu-QuaVZ-1}}\nwenddeflinemarkup
for (int i = 1; i < p; i++) \{
  // Extract vertices
  final int u = bnew[(3*i - 2)];
  final int v = bnew[(3*i + 1)];
  // Compute path and store into legs
  leg = this.tools.computeRoute(u, v, t);
  legs[i] = leg;
  \LA{}Debug: set leg~{\nwtagstyle{}\subpageref{NW8tIqu-2dVika-1}}\RA{}
  // Update n and t
  n += (leg.length - 2);
  t = leg[leg.length - 2];
  \LA{}Debug: set route length and time~{\nwtagstyle{}\subpageref{NW8tIqu-3WOc2v-1}}\RA{}
\}
\nwused{\\{NW8tIqu-QuaVZ-1}}\nwendcode{}\nwbegindocs{6}\nwdocspar

\section{Step 3: Join}

After the paths have been computed, we can join them together into {\Tt{}wnew\nwendquote}.
Until now, we've added to {\Tt{}n\nwendquote} each path length as each path is computed.
When we updated {\Tt{}n\nwendquote}, we left out one waypoint to prevent double-counting,
for example paths $A\rightarrow B$ and $B\rightarrow C$ have a joined length
of three, not four. We can use {\Tt{}n\nwendquote} now to initialize the size of {\Tt{}wnew\nwendquote}.

Next, we simply loop through each of the paths in {\Tt{}legs\nwendquote}, copying each
element into {\Tt{}wnew\nwendquote}. We'll remember the last copy position using {\Tt{}k\nwendquote}.
Then we'll use {\Tt{}i\nwendquote} to slide over {\Tt{}legs\nwendquote} and {\Tt{}j\nwendquote} to slide through each
path, copying into {\Tt{}wnew[k]\nwendquote} as we go along. To avoid double-counting the
ends, we'll skip the last waypoint in each path, except if it is the last
path in {\Tt{}legs\nwendquote}.

\nwenddocs{}\nwbegincode{7}\sublabel{NW8tIqu-1OT4xQ-1}\nwmargintag{{\nwtagstyle{}\subpageref{NW8tIqu-1OT4xQ-1}}}\moddef{Shortest: Step 3: Join~{\nwtagstyle{}\subpageref{NW8tIqu-1OT4xQ-1}}}\endmoddef\nwstartdeflinemarkup\nwusesondefline{\\{NW8tIqu-QuaVZ-1}}\nwenddeflinemarkup
wnew = new int[n];
int k = 0;
for (int i = 0; i < legs.length; i++) \{
  int rend = (legs[i].length - (i == (legs.length - 1) ? 0 : 2));
  for (int j = 0; j < rend; j++) \{
    wnew[k] = legs[i][j];
    \LA{}Debug: join route components~{\nwtagstyle{}\subpageref{NW8tIqu-J6KjD-1}}\RA{}
    k++;
  \}
\}
\nwused{\\{NW8tIqu-QuaVZ-1}}\nwendcode{}\nwbegindocs{8}\nwdocspar

\section{Step 4: Fill}

Finally, we can go back and fill in the event times into {\Tt{}bnew\nwendquote}. As each path
travels from one event to the next, all we need to do is extract the time
component of the first waypoint of each path, located in the 0th position of
the path, and copy it into {\Tt{}bnew\nwendquote}. The second path in {\Tt{}legs\nwendquote} starts from
the first event, so we'll initialize our {\Tt{}i\nwendquote} counter to one instead of zero.
Then when we copy into {\Tt{}bnew\nwendquote}, we'll shift the position to the left by three
elements because the first event in {\Tt{}bnew\nwendquote} starts from position 0.

\nwenddocs{}\nwbegincode{9}\sublabel{NW8tIqu-1nl6rv-1}\nwmargintag{{\nwtagstyle{}\subpageref{NW8tIqu-1nl6rv-1}}}\moddef{Shortest: Step 4: Fill~{\nwtagstyle{}\subpageref{NW8tIqu-1nl6rv-1}}}\endmoddef\nwstartdeflinemarkup\nwusesondefline{\\{NW8tIqu-QuaVZ-1}}\nwenddeflinemarkup
for (int i = 1; i < legs.length; i++) \{
  bnew[(3*i - 3)] = legs[i][0];
  \LA{}Debug: set schedule time~{\nwtagstyle{}\subpageref{NW8tIqu-1EJRY4-1}}\RA{}
\}
bnew[(3*p - 3)] = t;
\LA{}Debug: set vehicle end time~{\nwtagstyle{}\subpageref{NW8tIqu-3ZHBo5-1}}\RA{}
\nwused{\\{NW8tIqu-QuaVZ-1}}\nwendcode{}\nwbegindocs{10}\nwdocspar

\section{Full Code}

Here is the full code.

\begin{verbatim}
{
  final int p = (bnew.length/3);
  final int[][] legs = new int[p][];

  int[] leg = this.tools.computeRoute(wbeg[1], bnew[1], wbeg[0]);
  int n = leg.length;
  int t = leg[(n - 2)];

  legs[0] = leg;
  for (int i = 1; i < p; i++) {
    // Extract vertices
    final int u = bnew[(3*i - 2)];
    final int v = bnew[(3*i + 1)];
    // Compute path and store into legs
    leg = this.tools.computeRoute(u, v, t);
    legs[i] = leg;
    // Update n and t
    n += (leg.length - 2);
    t = leg[leg.length - 2];
  }
  wnew = new int[n];
  int k = 0;
  for (int i = 0; i < legs.length; i++) {
    int rend = (legs[i].length - (i == (legs.length - 1) ? 0 : 2));
    for (int j = 0; j < rend; j++) {
      wnew[k] = legs[i][j];
      k++;
    }
  }
  for (int i = 1; i < legs.length; i++) {
    bnew[(3*i - 3)] = legs[i][0];
  }
  bnew[(3*p - 3)] = t;
}
\end{verbatim}

\section{Debug}

\subsection{Debug: Echo Shortest Path}

\nwenddocs{}\nwbegincode{11}\sublabel{NW8tIqu-4ItA5T-1}\nwmargintag{{\nwtagstyle{}\subpageref{NW8tIqu-4ItA5T-1}}}\moddef{Debug: echo shortest path~{\nwtagstyle{}\subpageref{NW8tIqu-4ItA5T-1}}}\endmoddef\nwstartdeflinemarkup\nwusesondefline{\\{NW8tIqu-QuaVZ-1}}\nwenddeflinemarkup
if (DEBUG) \{
  System.out.printf("shortest path\\n");
\}
\nwused{\\{NW8tIqu-QuaVZ-1}}\nwendcode{}\nwbegindocs{12}\nwdocspar

\subsection{Debug: Initialize Legs Container}

\nwenddocs{}\nwbegincode{13}\sublabel{NW8tIqu-3R3yVH-1}\nwmargintag{{\nwtagstyle{}\subpageref{NW8tIqu-3R3yVH-1}}}\moddef{Debug: initialize legs container~{\nwtagstyle{}\subpageref{NW8tIqu-3R3yVH-1}}}\endmoddef\nwstartdeflinemarkup\nwusesondefline{\\{NW8tIqu-3lMPNM-1}}\nwenddeflinemarkup
if (DEBUG) \{
  System.out.printf("init legs=\{ \}\\n");
\}
\nwused{\\{NW8tIqu-3lMPNM-1}}\nwendcode{}\nwbegindocs{14}\nwdocspar

\subsection{Debug: Set Route Length and Time}

\nwenddocs{}\nwbegincode{15}\sublabel{NW8tIqu-3WOc2v-1}\nwmargintag{{\nwtagstyle{}\subpageref{NW8tIqu-3WOc2v-1}}}\moddef{Debug: set route length and time~{\nwtagstyle{}\subpageref{NW8tIqu-3WOc2v-1}}}\endmoddef\nwstartdeflinemarkup\nwusesondefline{\\{NW8tIqu-3lMPNM-1}\\{NW8tIqu-2EZ2CY-1}}\nwenddeflinemarkup
if (DEBUG) \{
  System.out.printf("set n=%d\\n", n);
  System.out.printf("set t=%d\\n", t);
\}
\nwused{\\{NW8tIqu-3lMPNM-1}\\{NW8tIqu-2EZ2CY-1}}\nwendcode{}\nwbegindocs{16}\nwdocspar

\subsection{Debug: Set First Leg}

\nwenddocs{}\nwbegincode{17}\sublabel{NW8tIqu-1F8aiy-1}\nwmargintag{{\nwtagstyle{}\subpageref{NW8tIqu-1F8aiy-1}}}\moddef{Debug: set first leg~{\nwtagstyle{}\subpageref{NW8tIqu-1F8aiy-1}}}\endmoddef\nwstartdeflinemarkup\nwusesondefline{\\{NW8tIqu-3lMPNM-1}}\nwenddeflinemarkup
if (DEBUG) \{
  System.out.printf("set legs[0]=\{ %d, %d, ..., %d, %d  \}\\n",
      legs[0][0], legs[0][1], legs[0][legs[0].length - 2], legs[0][legs[0].length - 1]);
\}
\nwused{\\{NW8tIqu-3lMPNM-1}}\nwendcode{}\nwbegindocs{18}\nwdocspar

\subsection{Debug: Set Leg}

\nwenddocs{}\nwbegincode{19}\sublabel{NW8tIqu-2dVika-1}\nwmargintag{{\nwtagstyle{}\subpageref{NW8tIqu-2dVika-1}}}\moddef{Debug: set leg~{\nwtagstyle{}\subpageref{NW8tIqu-2dVika-1}}}\endmoddef\nwstartdeflinemarkup\nwusesondefline{\\{NW8tIqu-2EZ2CY-1}}\nwenddeflinemarkup
if (DEBUG) \{
  System.out.printf("set legs[%d]=\{ %d, %d, ..., %d, %d  \}\\n", i,
      legs[i][0], legs[i][1], legs[i][legs[i].length - 2], legs[i][legs[i].length - 1]);
\}
\nwused{\\{NW8tIqu-2EZ2CY-1}}\nwendcode{}\nwbegindocs{20}\nwdocspar

\subsection{Debug: Join Route Components}

\nwenddocs{}\nwbegincode{21}\sublabel{NW8tIqu-J6KjD-1}\nwmargintag{{\nwtagstyle{}\subpageref{NW8tIqu-J6KjD-1}}}\moddef{Debug: join route components~{\nwtagstyle{}\subpageref{NW8tIqu-J6KjD-1}}}\endmoddef\nwstartdeflinemarkup\nwusesondefline{\\{NW8tIqu-1OT4xQ-1}}\nwenddeflinemarkup
if (DEBUG) \{
  System.out.printf("set wnew[%d]=%d\\n", k, wnew[k]);
\}
\nwused{\\{NW8tIqu-1OT4xQ-1}}\nwendcode{}\nwbegindocs{22}\nwdocspar

\subsection{Debug: Set Schedule Time}

\nwenddocs{}\nwbegincode{23}\sublabel{NW8tIqu-1EJRY4-1}\nwmargintag{{\nwtagstyle{}\subpageref{NW8tIqu-1EJRY4-1}}}\moddef{Debug: set schedule time~{\nwtagstyle{}\subpageref{NW8tIqu-1EJRY4-1}}}\endmoddef\nwstartdeflinemarkup\nwusesondefline{\\{NW8tIqu-1nl6rv-1}}\nwenddeflinemarkup
if (DEBUG) \{
  System.out.printf("set bnew[%d]=%d\\n", (3*i - 3), bnew[(3*i - 3)]);
\}
\nwused{\\{NW8tIqu-1nl6rv-1}}\nwendcode{}\nwbegindocs{24}\nwdocspar

\subsection{Debug: Set Vehicle End Time}

\nwenddocs{}\nwbegincode{25}\sublabel{NW8tIqu-3ZHBo5-1}\nwmargintag{{\nwtagstyle{}\subpageref{NW8tIqu-3ZHBo5-1}}}\moddef{Debug: set vehicle end time~{\nwtagstyle{}\subpageref{NW8tIqu-3ZHBo5-1}}}\endmoddef\nwstartdeflinemarkup\nwusesondefline{\\{NW8tIqu-1nl6rv-1}}\nwenddeflinemarkup
if (DEBUG) \{
  System.out.printf("set bnew[%d]=%d\\n", (3*p - 3), bnew[(3*p - 3)]);
\}
\nwused{\\{NW8tIqu-1nl6rv-1}}\nwendcode{}\nwbegindocs{26}\nwdocspar

\nwenddocs{}\nwfilename{src/client-overview.nw}\nwbegindocs{0}\part{Client Algorithms}
\label{part-client}

\chapter{Overview}
\label{client-overview}

This chapter lists an overview of the example client algorithms. Go to the page
next to each method name to go directly to the method definition.
Tables~\ref{tab:search-algorithms}~and~\ref{tab:join-algorithms} summarize some
features of the algorithms. Table~\ref{tab:performance-summary} summarize some
performance metrics measured using the {\Tt{}data/small\nwendquote} problem instance in the
Jargo root directory. Measurements were made using sequential mode.

\begin{table}[h]
\begin{center}
\caption{Summary of Search-Based Algorithms}
\label{tab:search-algorithms}
\begin{tabular}{|l|l|l|l|l|}
\hline
Ch. & Algorithm & Selection Predicate & Scheduling Strategy & Routing Strategy \\
\hline
Ch.~\ref{client-nearest-a} & Nearest A & Minimum Proximity & Prepend & Shortest Path \\
Ch.~\ref{client-nearest-b} & Nearest B & Minimum Proximity & Prepend After& Shortest Path \\
Ch.~\ref{client-greedy} & Greedy Insertion & Minimum Routing Cost & Insertion Heuristic & Shortest Path \\
\hline
\end{tabular}
\end{center}
\end{table}

\begin{table}[h]
\begin{center}
\caption{Summary of Join-Based Algorithms (TODO: Implement some join algorithms)}
\label{tab:join-algorithms}
\begin{tabular}{|l|l|l|l|l|}
\hline
Ch. & Algorithm & Join Strategy & Scheduling Strategy & Routing Strategy \\
\hline
\end{tabular}
\end{center}
\end{table}

\begin{table}[h]
\begin{center}
\caption{Performance Summary (\texttt{data/small}; Sequential Mode)}
\label{tab:performance-summary}
\begin{tabular}{|l|l|r|r|}
\hline
Ch. & Algorithm & Service Rate & Distance Ratio \\
\hline
Ch.~\ref{client-nearest-a} & Nearest A & 87.59\% & 1.47 \\
\hline
\end{tabular}
\end{center}
\end{table}

For Table~\ref{tab:performance-summary}, the columns are defined as:

\begin{itemize}
\item Service Rate: percentage of serviced requests over all requests
\item ...
\end{itemize}


\nwenddocs{}\nwfilename{src/client-nearest-a.nw}\nwbegindocs{0}\chapter{Client: Nearest A}
\label{client-nearest-a}

\nwenddocs{}\nwbegincode{1}\sublabel{NW2surku-zKRYu-1}\nwmargintag{{\nwtagstyle{}\subpageref{NW2surku-zKRYu-1}}}\moddef{NearestA.java~{\nwtagstyle{}\subpageref{NW2surku-zKRYu-1}}}\endmoddef\nwstartdeflinemarkup\nwenddeflinemarkup
\LA{}NearestA: preamble~{\nwtagstyle{}\subpageref{NW2surku-3wQFt9-1}}\RA{}
public class NearestA extends Client \{
  \LA{}NearestA: member variables~{\nwtagstyle{}\subpageref{NW2surku-2qcVD-1}}\RA{}
  protected \LA{}NearestA: handleRequest(1)~{\nwtagstyle{}\subpageref{NW2surku-4MdRSG-1}}\RA{}
  protected \LA{}NearestA: handleServerLocation(1)~{\nwtagstyle{}\subpageref{NW2surku-1xARKZ-1}}\RA{}
\}
\nwnotused{NearestA.java}\nwendcode{}\nwbegindocs{2}\nwdocspar

\nwenddocs{}\nwbegincode{3}\sublabel{NW2surku-3wQFt9-1}\nwmargintag{{\nwtagstyle{}\subpageref{NW2surku-3wQFt9-1}}}\moddef{NearestA: preamble~{\nwtagstyle{}\subpageref{NW2surku-3wQFt9-1}}}\endmoddef\nwstartdeflinemarkup\nwusesondefline{\\{NW2surku-zKRYu-1}\\{NW2rgQ87-105L3q-1}}\nwenddeflinemarkup
package com.github.jargors.client;
import com.github.jargors.sim.*;
import java.util.Map;
import java.util.Map.Entry;
import java.util.HashMap;
import java.util.concurrent.ConcurrentHashMap;
\nwused{\\{NW2surku-zKRYu-1}\\{NW2rgQ87-105L3q-1}}\nwendcode{}\nwbegindocs{4}\nwdocspar

\nwenddocs{}\nwbegincode{5}\sublabel{NW2surku-2qcVD-1}\nwmargintag{{\nwtagstyle{}\subpageref{NW2surku-2qcVD-1}}}\moddef{NearestA: member variables~{\nwtagstyle{}\subpageref{NW2surku-2qcVD-1}}}\endmoddef\nwstartdeflinemarkup\nwusesondefline{\\{NW2surku-zKRYu-1}\\{NW2rgQ87-105L3q-1}}\nwenddeflinemarkup
final int MAX_DELTA = 300;  // seconds from last vehicle location update
final int MAX_RANGE = 600;  // meters from last vehicle position
final ConcurrentHashMap<Integer, int[]> locations =
  new ConcurrentHashMap<Integer, int[]>();
\nwused{\\{NW2surku-zKRYu-1}\\{NW2rgQ87-105L3q-1}}\nwendcode{}\nwbegindocs{6}\nwdocspar

\nwenddocs{}\nwbegincode{7}\sublabel{NW2surku-4MdRSG-1}\nwmargintag{{\nwtagstyle{}\subpageref{NW2surku-4MdRSG-1}}}\moddef{NearestA: handleRequest(1)~{\nwtagstyle{}\subpageref{NW2surku-4MdRSG-1}}}\endmoddef\nwstartdeflinemarkup\nwusesondefline{\\{NW2surku-zKRYu-1}}\nwenddeflinemarkup
void handleRequest(int[] r) throws ClientException, ClientFatalException \{
  \LA{}Debug: handle request~{\nwtagstyle{}\subpageref{NW2surku-33A3UY-1}}\RA{}
  try \{
    \LA{}NearestA: Step 1: Initialize~{\nwtagstyle{}\subpageref{NW2surku-12kOmE-1}}\RA{}
    \LA{}NearestA: Step 2: Map and Filter~{\nwtagstyle{}\subpageref{NW2surku-1OsYBb-1}}\RA{}
    while (\LA{}NearestA: Go to Reduce?~{\nwtagstyle{}\subpageref{NW2surku-MbHJM-1}}\RA{}) \{
      \LA{}NearestA: Step 3: Reduce~{\nwtagstyle{}\subpageref{NW2surku-i3BV8-1}}\RA{}
      \LA{}NearestA: Step 4: Prepare~{\nwtagstyle{}\subpageref{NW2surku-lZU7g-1}}\RA{}
      \LA{}NearestA: Step 5: Schedule~{\nwtagstyle{}\subpageref{NW2surku-2iOiaE-1}}\RA{}
      if (\LA{}NearestA: Go to Submit?~{\nwtagstyle{}\subpageref{NW2surku-35BBt9-1}}\RA{}) \{
        \LA{}NearestA: Step 6: Route~{\nwtagstyle{}\subpageref{NW2surku-dWW2r-1}}\RA{}
        \LA{}NearestA: Step 7: Submit~{\nwtagstyle{}\subpageref{NW2surku-iFIwh-1}}\RA{}
      \} else \{
        \LA{}NearestA: Step 8: Reset~{\nwtagstyle{}\subpageref{NW2surku-3xagbK-1}}\RA{}
      \}
    \}
  \} catch (Exception e) \{
    throw new ClientException(e);
  \}
\}
\nwused{\\{NW2surku-zKRYu-1}}\nwendcode{}\nwbegindocs{8}\nwdocspar

\nwenddocs{}\nwbegincode{9}\sublabel{NW2surku-1xARKZ-1}\nwmargintag{{\nwtagstyle{}\subpageref{NW2surku-1xARKZ-1}}}\moddef{NearestA: handleServerLocation(1)~{\nwtagstyle{}\subpageref{NW2surku-1xARKZ-1}}}\endmoddef\nwstartdeflinemarkup\nwusesondefline{\\{NW2surku-zKRYu-1}\\{NW2rgQ87-105L3q-1}}\nwenddeflinemarkup
void handleServerLocation(int[] s) \{
  this.locations.put(s[0], new int[] \{ s[1], s[2] \});
  \LA{}Debug: put server location~{\nwtagstyle{}\subpageref{NW2surku-hcsfv-1}}\RA{}
\}
\nwused{\\{NW2surku-zKRYu-1}\\{NW2rgQ87-105L3q-1}}\nwendcode{}\nwbegindocs{10}\nwdocspar

\section{Step 1: Initialize}

\nwenddocs{}\nwbegincode{11}\sublabel{NW2surku-12kOmE-1}\nwmargintag{{\nwtagstyle{}\subpageref{NW2surku-12kOmE-1}}}\moddef{NearestA: Step 1: Initialize~{\nwtagstyle{}\subpageref{NW2surku-12kOmE-1}}}\endmoddef\nwstartdeflinemarkup\nwusesondefline{\\{NW2surku-4MdRSG-1}\\{NW2rgQ87-1Jb08C-1}}\nwenddeflinemarkup
final Map<Integer, Integer> candidates = new HashMap<Integer, Integer>();
Entry<Integer, Integer> min = null;
final int now = this.communicator.retrieveClock();
final int rid = r[0];
final int rq  = r[1];
final int ro  = r[4];
final int rd  = r[5];
boolean ok = false;
\LA{}Debug: initialize NearestA~{\nwtagstyle{}\subpageref{NW2surku-m1x9e-1}}\RA{}
\nwused{\\{NW2surku-4MdRSG-1}\\{NW2rgQ87-1Jb08C-1}}\nwendcode{}\nwbegindocs{12}\nwdocspar

\section{Step 2: Map and Filter}

\nwenddocs{}\nwbegincode{13}\sublabel{NW2surku-1OsYBb-1}\nwmargintag{{\nwtagstyle{}\subpageref{NW2surku-1OsYBb-1}}}\moddef{NearestA: Step 2: Map and Filter~{\nwtagstyle{}\subpageref{NW2surku-1OsYBb-1}}}\endmoddef\nwstartdeflinemarkup\nwusesondefline{\\{NW2surku-4MdRSG-1}\\{NW2rgQ87-1Jb08C-1}}\nwenddeflinemarkup
for (final Integer sid : locations.keySet()) \{
  final int t = locations.get(sid)[0];
  final int v = locations.get(sid)[1];

  final boolean in_delta = (t >= now - MAX_DELTA);
  \LA{}Debug: delta filter~{\nwtagstyle{}\subpageref{NW2surku-2I20iB-1}}\RA{}
  if (!in_delta)
    continue;

  final int range = this.tools.computeHaversine(v, ro);
  final boolean in_range = (0 != range && range <= MAX_RANGE);
  \LA{}Debug: range filter~{\nwtagstyle{}\subpageref{NW2surku-4aukgD-1}}\RA{}
  if (!in_range)
    continue;

  candidates.put(sid, range);
  \LA{}Debug: put candidate~{\nwtagstyle{}\subpageref{NW2surku-1Lo26M-1}}\RA{}
\}
\nwused{\\{NW2surku-4MdRSG-1}\\{NW2rgQ87-1Jb08C-1}}\nwendcode{}\nwbegindocs{14}\nwdocspar

\section{Step 3: Reduce}

\nwenddocs{}\nwbegincode{15}\sublabel{NW2surku-i3BV8-1}\nwmargintag{{\nwtagstyle{}\subpageref{NW2surku-i3BV8-1}}}\moddef{NearestA: Step 3: Reduce~{\nwtagstyle{}\subpageref{NW2surku-i3BV8-1}}}\endmoddef\nwstartdeflinemarkup\nwusesondefline{\\{NW2surku-4MdRSG-1}\\{NW2rgQ87-1Jb08C-1}}\nwenddeflinemarkup
for (final Entry<Integer, Integer> entry : candidates.entrySet()) \{
  if (min == null || min.getValue() > entry.getValue()) \{
    min = entry;
    \LA{}Debug: find minimum-range candidate~{\nwtagstyle{}\subpageref{NW2surku-3aXcFM-1}}\RA{}
  \}
\}
\nwused{\\{NW2surku-4MdRSG-1}\\{NW2rgQ87-1Jb08C-1}}\nwendcode{}\nwbegindocs{16}\nwdocspar

\section{Step 4: Prepare}

\nwenddocs{}\nwbegincode{17}\sublabel{NW2surku-lZU7g-1}\nwmargintag{{\nwtagstyle{}\subpageref{NW2surku-lZU7g-1}}}\moddef{NearestA: Step 4: Prepare~{\nwtagstyle{}\subpageref{NW2surku-lZU7g-1}}}\endmoddef\nwstartdeflinemarkup\nwusesondefline{\\{NW2surku-4MdRSG-1}}\nwenddeflinemarkup
final int sid = min.getKey();
final int[] brem = this.communicator.queryServerScheduleRemaining(sid, now);
final int[] wact = this.communicator.queryServerRouteActive(sid);
int[] bnew = new int[] \{ \};
int[] wnew = new int[] \{ \};
\LA{}Debug: prepare scheduling and routing~{\nwtagstyle{}\subpageref{NW2surku-fxfvb-1}}\RA{}
\nwused{\\{NW2surku-4MdRSG-1}}\nwendcode{}\nwbegindocs{18}\nwdocspar

\section{Step 5: Schedule}

Default case (3) is to check capacity and no cleanup. In special case 1,
no need for check capacity. In special case 2, need to do cleanup.

Capacity check is different for case 2 and 3. For case 2, start from time of
first event. For case 3, start from current time.

\nwenddocs{}\nwbegincode{19}\sublabel{NW2surku-2iOiaE-1}\nwmargintag{{\nwtagstyle{}\subpageref{NW2surku-2iOiaE-1}}}\moddef{NearestA: Step 5: Schedule~{\nwtagstyle{}\subpageref{NW2surku-2iOiaE-1}}}\endmoddef\nwstartdeflinemarkup\nwusesondefline{\\{NW2surku-4MdRSG-1}}\nwenddeflinemarkup
if (brem[2] == sid) \{
  \LA{}Debug: echo NearestA case 1~{\nwtagstyle{}\subpageref{NW2surku-1KFBk4-1}}\RA{}
  \LA{}NearestA: Step 5: Case 1~{\nwtagstyle{}\subpageref{NW2surku-154Axj-1}}\RA{}
\} else \{
  if (brem[0] == wact[2]) \{
    \LA{}Debug: echo NearestA case 2~{\nwtagstyle{}\subpageref{NW2surku-3f0c4y-1}}\RA{}
    \LA{}NearestA: Step 5: Case 2~{\nwtagstyle{}\subpageref{NW2surku-3J1EB5-1}}\RA{}
  \} else \{
    \LA{}Debug: echo NearestA case 3~{\nwtagstyle{}\subpageref{NW2surku-vpgYW-1}}\RA{}
    \LA{}NearestA: Step 5: Case 3~{\nwtagstyle{}\subpageref{NW2surku-1CRkLP-1}}\RA{}
  \}
\}
\nwused{\\{NW2surku-4MdRSG-1}}\nwendcode{}\nwbegindocs{20}\nwdocspar

\subsection{Step 5: Case 1}

In Case 1, the next event is the server's own destination. Do Prepend.

\nwenddocs{}\nwbegincode{21}\sublabel{NW2surku-154Axj-1}\nwmargintag{{\nwtagstyle{}\subpageref{NW2surku-154Axj-1}}}\moddef{NearestA: Step 5: Case 1~{\nwtagstyle{}\subpageref{NW2surku-154Axj-1}}}\endmoddef\nwstartdeflinemarkup\nwusesondefline{\\{NW2surku-2iOiaE-1}}\nwenddeflinemarkup
ok = true;
\LA{}Debug: set ok~{\nwtagstyle{}\subpageref{NW2surku-2embq0-1}}\RA{}
\LA{}Scheduling: Prepend~{\nwtagstyle{}\subpageref{NW31whEh-3w7K97-1}}\RA{}
\nwused{\\{NW2surku-2iOiaE-1}}\nwendcode{}\nwbegindocs{22}\nwdocspar

\subsection{Step 5: Case 2}

In Cases 2 and 3, the next event is not the server's own destination. In Case
2, the server is already along an edge to the next event. In other words, the
next vertex it will visit is the event vertex. We must do Prepend After because
the server's current course to the next event cannot be altered (see Limitation
\#4). Afterward, we must remove this next event from the newly constructed
schedule (see Limitation \#5).

\nwenddocs{}\nwbegincode{23}\sublabel{NW2surku-3J1EB5-1}\nwmargintag{{\nwtagstyle{}\subpageref{NW2surku-3J1EB5-1}}}\moddef{NearestA: Step 5: Case 2~{\nwtagstyle{}\subpageref{NW2surku-3J1EB5-1}}}\endmoddef\nwstartdeflinemarkup\nwusesondefline{\\{NW2surku-2iOiaE-1}}\nwenddeflinemarkup
ok = (this.communicator.queryServerCapacityViolations(
    sid, rq, brem[0], brem[4])[0] == 0);
\LA{}Debug: set ok~{\nwtagstyle{}\subpageref{NW2surku-2embq0-1}}\RA{}
if (ok) \{
  \LA{}Scheduling: Prepend After~{\nwtagstyle{}\subpageref{NW1UKIGb-3ujU2A-1}}\RA{}
  int[] temp = new int[(bnew.length - 3)];
  for (int i = 0; i < temp.length; i++) \{
    temp[i] = bnew[(i + 3)];
  \}
  bnew = temp;
\}
\nwused{\\{NW2surku-2iOiaE-1}}\nwendcode{}\nwbegindocs{24}\nwdocspar

\subsection{Step 5: Case 3}

In Case 3, the next vertex the server will visit is not the next event and
is not the server's own destination. Check capacity, then do Prepend.

\nwenddocs{}\nwbegincode{25}\sublabel{NW2surku-1CRkLP-1}\nwmargintag{{\nwtagstyle{}\subpageref{NW2surku-1CRkLP-1}}}\moddef{NearestA: Step 5: Case 3~{\nwtagstyle{}\subpageref{NW2surku-1CRkLP-1}}}\endmoddef\nwstartdeflinemarkup\nwusesondefline{\\{NW2surku-2iOiaE-1}}\nwenddeflinemarkup
ok = (this.communicator.queryServerCapacityViolations(
    sid, rq, now, brem[0])[0] == 0);
\LA{}Debug: set ok~{\nwtagstyle{}\subpageref{NW2surku-2embq0-1}}\RA{}
if (ok) \{
  \LA{}Scheduling: Prepend~{\nwtagstyle{}\subpageref{NW31whEh-3w7K97-1}}\RA{}
\}
\nwused{\\{NW2surku-2iOiaE-1}}\nwendcode{}\nwbegindocs{26}\nwdocspar

\section{Step 6: Route}

\nwenddocs{}\nwbegincode{27}\sublabel{NW2surku-dWW2r-1}\nwmargintag{{\nwtagstyle{}\subpageref{NW2surku-dWW2r-1}}}\moddef{NearestA: Step 6: Route~{\nwtagstyle{}\subpageref{NW2surku-dWW2r-1}}}\endmoddef\nwstartdeflinemarkup\nwusesondefline{\\{NW2surku-4MdRSG-1}}\nwenddeflinemarkup
// start route from next waypoint (not last-visited)
final int[] wbeg = new int[] \{ wact[2], wact[3] \};
if (wact[3] == 0) \{
  // if next waypoint is vehicle destination, then start from (now, last-visited)
  wbeg[0] = now;
  wbeg[1] = wact[1];
\}
\LA{}Debug: set initial waypoint~{\nwtagstyle{}\subpageref{NW2surku-1GjT93-1}}\RA{}
\LA{}Routing: Shortest Path~{\nwtagstyle{}\subpageref{NW8tIqu-QuaVZ-1}}\RA{}
if (wact[3] == 0) \{
  // if next waypoint is vehicle destination, reset route start time to last-visited time
  wnew[0] = locations.get(sid)[0];
\}
\LA{}Debug: finish routing~{\nwtagstyle{}\subpageref{NW2surku-28JmJw-1}}\RA{}
\nwused{\\{NW2surku-4MdRSG-1}}\nwendcode{}\nwbegindocs{28}\nwdocspar

\section{Step 7: Submit}

\nwenddocs{}\nwbegincode{29}\sublabel{NW2surku-iFIwh-1}\nwmargintag{{\nwtagstyle{}\subpageref{NW2surku-iFIwh-1}}}\moddef{NearestA: Step 7: Submit~{\nwtagstyle{}\subpageref{NW2surku-iFIwh-1}}}\endmoddef\nwstartdeflinemarkup\nwusesondefline{\\{NW2surku-4MdRSG-1}}\nwenddeflinemarkup
this.communicator.updateServerService(sid, wnew, bnew,
    new int[] \{ rid \}, new int[] \{ \});
\LA{}Debug: submit assignment~{\nwtagstyle{}\subpageref{NW2surku-2uYsdK-1}}\RA{}
\nwused{\\{NW2surku-4MdRSG-1}}\nwendcode{}\nwbegindocs{30}\nwdocspar

\section{Step 8: Reset}

\nwenddocs{}\nwbegincode{31}\sublabel{NW2surku-3xagbK-1}\nwmargintag{{\nwtagstyle{}\subpageref{NW2surku-3xagbK-1}}}\moddef{NearestA: Step 8: Reset~{\nwtagstyle{}\subpageref{NW2surku-3xagbK-1}}}\endmoddef\nwstartdeflinemarkup\nwusesondefline{\\{NW2surku-4MdRSG-1}}\nwenddeflinemarkup
min = null;
candidates.remove(sid);
\LA{}Debug: reset~{\nwtagstyle{}\subpageref{NW2surku-1iDlmC-1}}\RA{}
\nwused{\\{NW2surku-4MdRSG-1}}\nwendcode{}\nwbegindocs{32}\nwdocspar

\section{Conditions for Continuing}

\nwenddocs{}\nwbegincode{33}\sublabel{NW2surku-MbHJM-1}\nwmargintag{{\nwtagstyle{}\subpageref{NW2surku-MbHJM-1}}}\moddef{NearestA: Go to Reduce?~{\nwtagstyle{}\subpageref{NW2surku-MbHJM-1}}}\endmoddef\nwstartdeflinemarkup\nwusesondefline{\\{NW2surku-4MdRSG-1}}\nwenddeflinemarkup
ok == false && candidates.size() > 0
\nwused{\\{NW2surku-4MdRSG-1}}\nwendcode{}\nwbegindocs{34}\nwdocspar

\nwenddocs{}\nwbegincode{35}\sublabel{NW2surku-35BBt9-1}\nwmargintag{{\nwtagstyle{}\subpageref{NW2surku-35BBt9-1}}}\moddef{NearestA: Go to Submit?~{\nwtagstyle{}\subpageref{NW2surku-35BBt9-1}}}\endmoddef\nwstartdeflinemarkup\nwusesondefline{\\{NW2surku-4MdRSG-1}}\nwenddeflinemarkup
ok == true
\nwused{\\{NW2surku-4MdRSG-1}}\nwendcode{}\nwbegindocs{36}\nwdocspar

\section{Debug}

\subsection{Debug: Handle Request}

\nwenddocs{}\nwbegincode{37}\sublabel{NW2surku-33A3UY-1}\nwmargintag{{\nwtagstyle{}\subpageref{NW2surku-33A3UY-1}}}\moddef{Debug: handle request~{\nwtagstyle{}\subpageref{NW2surku-33A3UY-1}}}\endmoddef\nwstartdeflinemarkup\nwusesondefline{\\{NW2surku-4MdRSG-1}\\{NW2rgQ87-1Jb08C-1}}\nwenddeflinemarkup
if (DEBUG) \{
  System.out.printf("got request=\{ id=%d, q=%d, e=%d, l=%d, o=%d, d=%d, b=%d \}\\n",
      r[0], r[1], r[2], r[3], r[4], r[5], r[6]);
\}
\nwused{\\{NW2surku-4MdRSG-1}\\{NW2rgQ87-1Jb08C-1}}\nwendcode{}\nwbegindocs{38}\nwdocspar

\subsection{Debug: Initialize NearestA}

\nwenddocs{}\nwbegincode{39}\sublabel{NW2surku-m1x9e-1}\nwmargintag{{\nwtagstyle{}\subpageref{NW2surku-m1x9e-1}}}\moddef{Debug: initialize NearestA~{\nwtagstyle{}\subpageref{NW2surku-m1x9e-1}}}\endmoddef\nwstartdeflinemarkup\nwusesondefline{\\{NW2surku-12kOmE-1}}\nwenddeflinemarkup
if (DEBUG) \{
  System.out.printf("got now=%d\\n", now);
  System.out.printf("init candidates[]=\{ \}\\n");
  System.out.printf("init min=null\\n");
  System.out.printf("init ok=false\\n");
\}
\nwused{\\{NW2surku-12kOmE-1}}\nwendcode{}\nwbegindocs{40}\nwdocspar

\subsection{Debug: Delta Filter}

\nwenddocs{}\nwbegincode{41}\sublabel{NW2surku-2I20iB-1}\nwmargintag{{\nwtagstyle{}\subpageref{NW2surku-2I20iB-1}}}\moddef{Debug: delta filter~{\nwtagstyle{}\subpageref{NW2surku-2I20iB-1}}}\endmoddef\nwstartdeflinemarkup\nwusesondefline{\\{NW2surku-1OsYBb-1}}\nwenddeflinemarkup
if (DEBUG) \{
  System.out.printf("got %d in_delta=%s\\n", sid, (in_delta ? "true" : "false"));
\}
\nwused{\\{NW2surku-1OsYBb-1}}\nwendcode{}\nwbegindocs{42}\nwdocspar

\subsection{Debug: Range Filter}

\nwenddocs{}\nwbegincode{43}\sublabel{NW2surku-4aukgD-1}\nwmargintag{{\nwtagstyle{}\subpageref{NW2surku-4aukgD-1}}}\moddef{Debug: range filter~{\nwtagstyle{}\subpageref{NW2surku-4aukgD-1}}}\endmoddef\nwstartdeflinemarkup\nwusesondefline{\\{NW2surku-1OsYBb-1}}\nwenddeflinemarkup
if (DEBUG) \{
  System.out.printf("got %d in_range=%s\\n", sid, (in_range ? "true" : "false"));
\}
\nwused{\\{NW2surku-1OsYBb-1}}\nwendcode{}\nwbegindocs{44}\nwdocspar

\subsection{Debug: Put Candidate}

\nwenddocs{}\nwbegincode{45}\sublabel{NW2surku-1Lo26M-1}\nwmargintag{{\nwtagstyle{}\subpageref{NW2surku-1Lo26M-1}}}\moddef{Debug: put candidate~{\nwtagstyle{}\subpageref{NW2surku-1Lo26M-1}}}\endmoddef\nwstartdeflinemarkup\nwusesondefline{\\{NW2surku-1OsYBb-1}}\nwenddeflinemarkup
if (DEBUG) \{
  System.out.printf("put candidates[], key=%d, val=%d\\n", sid, range);
\}
\nwused{\\{NW2surku-1OsYBb-1}}\nwendcode{}\nwbegindocs{46}\nwdocspar

\subsection{Debug: Find Minimum-Range Candidate}

\nwenddocs{}\nwbegincode{47}\sublabel{NW2surku-3aXcFM-1}\nwmargintag{{\nwtagstyle{}\subpageref{NW2surku-3aXcFM-1}}}\moddef{Debug: find minimum-range candidate~{\nwtagstyle{}\subpageref{NW2surku-3aXcFM-1}}}\endmoddef\nwstartdeflinemarkup\nwusesondefline{\\{NW2surku-i3BV8-1}}\nwenddeflinemarkup
if (DEBUG) \{
  System.out.printf("set min=\{ %d, %d \}\\n", min.getKey(), min.getValue());
\}
\nwused{\\{NW2surku-i3BV8-1}}\nwendcode{}\nwbegindocs{48}\nwdocspar

\subsection{Debug: Prepare Scheduling and Routing}
\nwenddocs{}\nwbegincode{49}\sublabel{NW2surku-fxfvb-1}\nwmargintag{{\nwtagstyle{}\subpageref{NW2surku-fxfvb-1}}}\moddef{Debug: prepare scheduling and routing~{\nwtagstyle{}\subpageref{NW2surku-fxfvb-1}}}\endmoddef\nwstartdeflinemarkup\nwusesondefline{\\{NW2surku-lZU7g-1}}\nwenddeflinemarkup
if (DEBUG) \{
  System.out.printf("got brem: \\n");
  for (int i = 0; i < (brem.length - 3); i+=4) \{
    System.out.printf("  \{ t=%d, v=%d, ls=%d, lr=%d \}\\n",
        brem[i], brem[i+1], brem[i+2], brem[i+3]);
  \}
  System.out.printf("got wact: \\n");
  for (int i = 0; i < (wact.length - 1); i += 2) \{
    System.out.printf("  \{ t=%d, v=%d \},\\n",
      wact[i], wact[i+1]);
  \}
\}
\nwused{\\{NW2surku-lZU7g-1}}\nwendcode{}\nwbegindocs{50}\nwdocspar

\subsection{Debug: Echo NearestA Case 1}

\nwenddocs{}\nwbegincode{51}\sublabel{NW2surku-1KFBk4-1}\nwmargintag{{\nwtagstyle{}\subpageref{NW2surku-1KFBk4-1}}}\moddef{Debug: echo NearestA case 1~{\nwtagstyle{}\subpageref{NW2surku-1KFBk4-1}}}\endmoddef\nwstartdeflinemarkup\nwusesondefline{\\{NW2surku-2iOiaE-1}}\nwenddeflinemarkup
if (DEBUG) \{
  System.out.printf("goto case 1\\n");
\}
\nwused{\\{NW2surku-2iOiaE-1}}\nwendcode{}\nwbegindocs{52}\nwdocspar

\subsection{Debug: Echo NearestA Case 2}

\nwenddocs{}\nwbegincode{53}\sublabel{NW2surku-3f0c4y-1}\nwmargintag{{\nwtagstyle{}\subpageref{NW2surku-3f0c4y-1}}}\moddef{Debug: echo NearestA case 2~{\nwtagstyle{}\subpageref{NW2surku-3f0c4y-1}}}\endmoddef\nwstartdeflinemarkup\nwusesondefline{\\{NW2surku-2iOiaE-1}}\nwenddeflinemarkup
if (DEBUG) \{
  System.out.printf("goto case 2\\n");
\}
\nwused{\\{NW2surku-2iOiaE-1}}\nwendcode{}\nwbegindocs{54}\nwdocspar

\subsection{Debug: Echo NearestA Case 3}

\nwenddocs{}\nwbegincode{55}\sublabel{NW2surku-vpgYW-1}\nwmargintag{{\nwtagstyle{}\subpageref{NW2surku-vpgYW-1}}}\moddef{Debug: echo NearestA case 3~{\nwtagstyle{}\subpageref{NW2surku-vpgYW-1}}}\endmoddef\nwstartdeflinemarkup\nwusesondefline{\\{NW2surku-2iOiaE-1}}\nwenddeflinemarkup
if (DEBUG) \{
  System.out.printf("goto case 3\\n");
\}
\nwused{\\{NW2surku-2iOiaE-1}}\nwendcode{}\nwbegindocs{56}\nwdocspar

\subsection{Debug: Set Ok}

\nwenddocs{}\nwbegincode{57}\sublabel{NW2surku-2embq0-1}\nwmargintag{{\nwtagstyle{}\subpageref{NW2surku-2embq0-1}}}\moddef{Debug: set ok~{\nwtagstyle{}\subpageref{NW2surku-2embq0-1}}}\endmoddef\nwstartdeflinemarkup\nwusesondefline{\\{NW2surku-154Axj-1}\\{NW2surku-3J1EB5-1}\\{NW2surku-1CRkLP-1}}\nwenddeflinemarkup
if (DEBUG) \{
  System.out.printf("set ok=%s\\n", ok ? "true" : "false");
\}
\nwused{\\{NW2surku-154Axj-1}\\{NW2surku-3J1EB5-1}\\{NW2surku-1CRkLP-1}}\nwendcode{}\nwbegindocs{58}\nwdocspar

\subsection{Debug: Set Initial Waypoint}

\nwenddocs{}\nwbegincode{59}\sublabel{NW2surku-1GjT93-1}\nwmargintag{{\nwtagstyle{}\subpageref{NW2surku-1GjT93-1}}}\moddef{Debug: set initial waypoint~{\nwtagstyle{}\subpageref{NW2surku-1GjT93-1}}}\endmoddef\nwstartdeflinemarkup\nwusesondefline{\\{NW2surku-dWW2r-1}}\nwenddeflinemarkup
if (DEBUG) \{
  System.out.printf("set wbeg[0..1]=\{%d, %d\}\\n", wbeg[0], wbeg[1]);
\}
\nwused{\\{NW2surku-dWW2r-1}}\nwendcode{}\nwbegindocs{60}\nwdocspar

\subsection{Debug: Finish Routing}

\nwenddocs{}\nwbegincode{61}\sublabel{NW2surku-28JmJw-1}\nwmargintag{{\nwtagstyle{}\subpageref{NW2surku-28JmJw-1}}}\moddef{Debug: finish routing~{\nwtagstyle{}\subpageref{NW2surku-28JmJw-1}}}\endmoddef\nwstartdeflinemarkup\nwusesondefline{\\{NW2surku-dWW2r-1}}\nwenddeflinemarkup
if (DEBUG) \{
  System.out.printf("set wnew[0]=%d\\n", wnew[0]);
\}
\nwused{\\{NW2surku-dWW2r-1}}\nwendcode{}\nwbegindocs{62}\nwdocspar

\subsection{Debug: Submit}

\nwenddocs{}\nwbegincode{63}\sublabel{NW2surku-2uYsdK-1}\nwmargintag{{\nwtagstyle{}\subpageref{NW2surku-2uYsdK-1}}}\moddef{Debug: submit assignment~{\nwtagstyle{}\subpageref{NW2surku-2uYsdK-1}}}\endmoddef\nwstartdeflinemarkup\nwusesondefline{\\{NW2surku-iFIwh-1}}\nwenddeflinemarkup
if (DEBUG) \{
  System.out.printf("submit:\\n");
  System.out.printf("  server=%d\\n", sid);
  System.out.printf("  wnew=\{ %d, %d, ..., %d, %d \}\\n",
      wnew[0], wnew[1], wnew[wnew.length - 2], wnew[wnew.length - 1]);
  System.out.printf("  bnew=\{ %d, %d, %d, ..., %d, %d, %d \}\\n",
      bnew[0], bnew[1], bnew[2],
      bnew[bnew.length - 3], bnew[bnew.length - 2], bnew[bnew.length - 1]);
  System.out.printf("  radd=\{ %d \}\\n", r[0]);
  System.out.printf("  rsub=\{ \}\\n");
\}
\nwused{\\{NW2surku-iFIwh-1}}\nwendcode{}\nwbegindocs{64}\nwdocspar

\subsection{Debug: Reset}

\nwenddocs{}\nwbegincode{65}\sublabel{NW2surku-1iDlmC-1}\nwmargintag{{\nwtagstyle{}\subpageref{NW2surku-1iDlmC-1}}}\moddef{Debug: reset~{\nwtagstyle{}\subpageref{NW2surku-1iDlmC-1}}}\endmoddef\nwstartdeflinemarkup\nwusesondefline{\\{NW2surku-3xagbK-1}}\nwenddeflinemarkup
if (DEBUG) \{
  System.out.printf("set min=null, remove candidate %d\\n", sid);
\}
\nwused{\\{NW2surku-3xagbK-1}}\nwendcode{}\nwbegindocs{66}\nwdocspar

\subsection{Debug: Put Server Location}

\nwenddocs{}\nwbegincode{67}\sublabel{NW2surku-hcsfv-1}\nwmargintag{{\nwtagstyle{}\subpageref{NW2surku-hcsfv-1}}}\moddef{Debug: put server location~{\nwtagstyle{}\subpageref{NW2surku-hcsfv-1}}}\endmoddef\nwstartdeflinemarkup\nwusesondefline{\\{NW2surku-1xARKZ-1}}\nwenddeflinemarkup
if (DEBUG) \{
  System.out.printf("put locations[%d]=[ %d, %d ]\\n", s[0], s[1], s[2]);
\}
\nwused{\\{NW2surku-1xARKZ-1}}\nwendcode{}\nwbegindocs{68}\nwdocspar

\section{Reference}

Here is the full NearestA.java code, with debug sections removed.

{
\small
\begin{verbatim}
package com.github.jargors.client;
import com.github.jargors.sim.*;
import java.util.Map;
import java.util.Map.Entry;
import java.util.HashMap;
import java.util.concurrent.ConcurrentHashMap;
public class NearestA extends Client {
  final int MAX_DELTA = 300;  // seconds from last vehicle location update
  final int MAX_RANGE = 600;  // meters from last vehicle position
  final ConcurrentHashMap<Integer, int[]> locations =
    new ConcurrentHashMap<Integer, int[]>();
  protected void handleRequest(int[] r) throws ClientException, ClientFatalException {
    final int rid = r[0];
    final int ro  = r[4];
    final int rd  = r[5];
    try {
      final int now = this.communicator.retrieveClock();

      final Map<Integer, Integer> candidates = new HashMap<Integer, Integer>();
      for (final Integer sid : locations.keySet()) {
        final int t = locations.get(sid)[0];
        final int v = locations.get(sid)[1];

        final boolean in_delta = (t >= now - MAX_DELTA);
        if (!in_delta)
          continue;

        final int range = this.tools.computeHaversine(v, ro);
        final boolean in_range = (0 != range && range <= MAX_RANGE);
        if (!in_range)
          continue;

        candidates.put(sid, range);
      }
      Entry<Integer, Integer> min = null;
      for (final Entry<Integer, Integer> entry : candidates.entrySet()) {
        if (min == null || min.getValue() > entry.getValue()) {
          min = entry;
        }
      }
      if (min != null) {
        final int sid = min.getKey();
        final int[] brem = this.communicator.queryServerScheduleRemaining(sid, now);
        final int m = (brem.length/4);
        final int p = (m + 2);

        final int[] bnew = new int[(3*p)];
        bnew[1] = ro;
        bnew[2] = rid;
        bnew[4] = rd;
        bnew[5] = rid;
        for (int i = 0; i < m; i++) {
          final int bv = brem[(4*i + 1)];
          final int ls = brem[(4*i + 2)];
          final int lr = brem[(4*i + 3)];
          bnew[(3*i + 7)] = bv;
          bnew[(3*i + 8)] = (ls == 0 ? lr : ls);
        }
        final int[][] legs = new int[p][];

        int[] leg = this.tools.computeRoute(locations.get(sid)[1], ro, now);
        int n = leg.length;
        int t = leg[n - 2];

        leg[0] = locations.get(sid)[0];
        legs[0] = leg;
        for (int i = 1; i < p; i++) {
          final int u = bnew[(3*i - 2)];
          final int v = bnew[(3*i + 1)];
          leg = this.tools.computeRoute(u, v, t);
          legs[i] = leg;

          n += (leg.length - 2);
          t = leg[leg.length - 2];
        }
        final int[] wnew = new int[n];
        {
          int k = 0;
          for (int i = 0; i < legs.length; i++) {
            int rend = (legs[i].length - (i == (legs.length - 1) ? 0 : 2));
            for (int j = 0; j < rend; j++) {
              wnew[k] = legs[i][j];
              k++;
            }
          }
        }
        for (int i = 1; i < legs.length; i++) {
          bnew[(3*i - 3)] = legs[i][0];
        }

        bnew[(3*p - 3)] = t;
        this.communicator.updateServerService(sid, wnew, bnew,
            new int[] { rid }, new int[] { });
      }
    } catch (Exception e) {
      throw new ClientException(e);
    }
  }
  protected void handleServerLocation(int[] s) {
    this.locations.put(s[0], new int[] { s[1], s[2] });
  }
}
\end{verbatim}
}

\nwenddocs{}\nwfilename{src/client-nearest-b.nw}\nwbegindocs{0}\chapter{Client: Nearest B}
\label{client-nearest-b}

The Nearest B algorithm uses prepend-after for scheduling. There are two
differences from Nearest A.

\nwenddocs{}\nwbegincode{1}\sublabel{NW2rgQ87-105L3q-1}\nwmargintag{{\nwtagstyle{}\subpageref{NW2rgQ87-105L3q-1}}}\moddef{NearestB.java~{\nwtagstyle{}\subpageref{NW2rgQ87-105L3q-1}}}\endmoddef\nwstartdeflinemarkup\nwenddeflinemarkup
\LA{}NearestA: preamble~{\nwtagstyle{}\subpageref{NW2surku-3wQFt9-1}}\RA{}
public class NearestB extends Client \{
  \LA{}NearestA: member variables~{\nwtagstyle{}\subpageref{NW2surku-2qcVD-1}}\RA{}
  protected \LA{}NearestB: handleRequest(1)~{\nwtagstyle{}\subpageref{NW2rgQ87-1Jb08C-1}}\RA{}
  protected \LA{}NearestA: handleServerLocation(1)~{\nwtagstyle{}\subpageref{NW2surku-1xARKZ-1}}\RA{}
\}
\nwnotused{NearestB.java}\nwendcode{}\nwbegindocs{2}\nwdocspar

\nwenddocs{}\nwbegincode{3}\sublabel{NW2rgQ87-1Jb08C-1}\nwmargintag{{\nwtagstyle{}\subpageref{NW2rgQ87-1Jb08C-1}}}\moddef{NearestB: handleRequest(1)~{\nwtagstyle{}\subpageref{NW2rgQ87-1Jb08C-1}}}\endmoddef\nwstartdeflinemarkup\nwusesondefline{\\{NW2rgQ87-105L3q-1}}\nwenddeflinemarkup
void handleRequest(int[] r) throws ClientException, ClientFatalException \{
  \LA{}Debug: handle request~{\nwtagstyle{}\subpageref{NW2surku-33A3UY-1}}\RA{}
  final int rid = r[0];
  final int ro  = r[4];
  final int rd  = r[5];
  try \{
    \LA{}NearestA: Step 1: Initialize~{\nwtagstyle{}\subpageref{NW2surku-12kOmE-1}}\RA{}
    \LA{}NearestA: Step 2: Map and Filter~{\nwtagstyle{}\subpageref{NW2surku-1OsYBb-1}}\RA{}
    \LA{}NearestA: Step 3: Reduce~{\nwtagstyle{}\subpageref{NW2surku-i3BV8-1}}\RA{}
    if (min != null) \{
      //NearestA: Step 4: Prepare Scheduling
      //Scheduling: Prepend After
      //NearestB: Step 5: Prepare Routing
      //Routing: Shortest Path
      //NearestA: Step 6: Submit
    \}
  \} catch (Exception e) \{
    throw new ClientException(e);
  \}
\}
\nwused{\\{NW2rgQ87-105L3q-1}}\nwendcode{}\nwbegindocs{4}\nwdocspar

For Nearest A, the time for the initial waypoint for the new route is at the
simulation world time. But because we do prepend-after here, we don't need to
recompute the first leg of the route. So we set the initial waypoint to the
next immediate event, if it exists. Otherwise, same as Nearest A.

\nwenddocs{}\nwbegincode{5}\sublabel{NW2rgQ87-My4EY-1}\nwmargintag{{\nwtagstyle{}\subpageref{NW2rgQ87-My4EY-1}}}\moddef{NearestB: Step 5: Prepare Routing~{\nwtagstyle{}\subpageref{NW2rgQ87-My4EY-1}}}\endmoddef\nwstartdeflinemarkup\nwenddeflinemarkup
\LA{}NearestA: Step 5: Prepare Routing~{\nwtagstyle{}\subpageref{nw@notdef}}\RA{}
if (m > 1) \{
    wbeg[0] = brem[0];
    wbeg[1] = brem[1];
\}
\nwnotused{NearestB: Step 5: Prepare Routing}\nwendcode{}\nwbegindocs{6}\nwdocspar

\section{Reference}

{
\small
\begin{verbatim}
package com.github.jargors.client;
import com.github.jargors.sim.*;
import java.util.Map;
import java.util.Map.Entry;
import java.util.HashMap;
import java.util.concurrent.ConcurrentHashMap;
public class NearestB extends Client {
  final int MAX_DELTA = 300;  // seconds from last vehicle location update
  final int MAX_RANGE = 600;  // meters from last vehicle position
  final ConcurrentHashMap<Integer, int[]> locations =
    new ConcurrentHashMap<Integer, int[]>();
  protected void handleRequest(int[] r) throws ClientException, ClientFatalException {
    final int rid = r[0];
    final int ro  = r[4];
    final int rd  = r[5];
    try {
      final int now = this.communicator.retrieveClock();

      final Map<Integer, Integer> candidates = new HashMap<Integer, Integer>();
      for (final Integer sid : locations.keySet()) {
        final int t = locations.get(sid)[0];
        final int v = locations.get(sid)[1];

        final boolean in_delta = (t >= now - MAX_DELTA);
        if (!in_delta)
          continue;

        final int range = this.tools.computeHaversine(v, ro);
        final boolean in_range = (0 != range && range <= MAX_RANGE);
        if (!in_range)
          continue;

        candidates.put(sid, range);
      }
      Entry<Integer, Integer> min = null;
      for (final Entry<Integer, Integer> entry : candidates.entrySet()) {
        if (min == null || min.getValue() > entry.getValue()) {
          min = entry;
        }
      }
      if (min != null) {
        final int sid = min.getKey();
        final int[] brem = this.communicator.queryServerScheduleRemaining(sid, now);
        final int m = (brem.length/4);
        final int p = (m + 2);

        final int[] bnew = new int[(3*p)];
        if (m == 1) {
          bnew[1] = ro;
          bnew[2] = rid;
          bnew[4] = rd;
          bnew[5] = rid;
          for (int i = 0; i < m; i++) {
            final int bv = brem[(4*i + 1)];
            final int ls = brem[(4*i + 2)];
            final int lr = brem[(4*i + 3)];
            bnew[(3*i + 7)] = bv;
            bnew[(3*i + 8)] = (ls == 0 ? lr : ls);
          }
        } else {
          bnew[4] = ro;
          bnew[5] = rid;
          bnew[7] = rd;
          bnew[8] = rid;
          for (int i = 0; i < m; i++) {
            final int bv = brem[(4*i + 1)];
            final int ls = brem[(4*i + 2)];
            final int lr = brem[(4*i + 3)];
            bnew[(3*i + (i < 1 ? 1 : 7))] = bv;
            bnew[(3*i + (i < 1 ? 2 : 8))] = (ls == 0 ? lr : ls);
          }
        }
        final int[][] legs = new int[p][];

        int[] leg = this.tools.computeRoute(locations.get(sid)[1], ro, now);
        int n = leg.length;
        int t = leg[n - 2];

        leg[0] = locations.get(sid)[0];
        legs[0] = leg;
        for (int i = 1; i < p; i++) {
          final int u = bnew[(3*i - 2)];
          final int v = bnew[(3*i + 1)];
          leg = this.tools.computeRoute(u, v, t);
          legs[i] = leg;

          n += (leg.length - 2);
          t = leg[leg.length - 2];
        }
        final int[] wnew = new int[n];
        {
          int k = 0;
          for (int i = 0; i < legs.length; i++) {
            int rend = (legs[i].length - (i == (legs.length - 1) ? 0 : 2));
            for (int j = 0; j < rend; j++) {
              wnew[k] = legs[i][j];
              k++;
            }
          }
        }
        for (int i = 1; i < legs.length; i++) {
          bnew[(3*i - 3)] = legs[i][0];
        }

        bnew[(3*p - 3)] = t;
        this.communicator.updateServerService(sid, wnew, bnew,
            new int[] { rid }, new int[] { });
      }
    } catch (Exception e) {
      throw new ClientException(e);
    }
  }
  protected void handleServerLocation(int[] s) {
    this.locations.put(s[0], new int[] { s[1], s[2] });
  }
}
\end{verbatim}
}


\nwenddocs{}\nwfilename{src/client-greedy.nw}\nwbegindocs{0}\chapter{Client: GreedyInsertion}
\label{client-greedy}

\nwenddocs{}\nwbegincode{1}\sublabel{NWkwPRG-1Tuf6f-1}\nwmargintag{{\nwtagstyle{}\subpageref{NWkwPRG-1Tuf6f-1}}}\moddef{GreedyInsertion.java~{\nwtagstyle{}\subpageref{NWkwPRG-1Tuf6f-1}}}\endmoddef\nwstartdeflinemarkup\nwenddeflinemarkup
\LA{}GreedyInsertion.java preamble~{\nwtagstyle{}\subpageref{NWkwPRG-1neuor-1}}\RA{}
public class GreedyInsertion extends Client \{
  \LA{}GreedyInsertion: member variables~{\nwtagstyle{}\subpageref{NWkwPRG-20xkUX-1}}\RA{}
  protected \LA{}GreedyInsertion: handleRequest(1)~{\nwtagstyle{}\subpageref{NWkwPRG-2OWvsM-1}}\RA{}
  protected \LA{}GreedyInsertion: endCollectServerLocations(1)~{\nwtagstyle{}\subpageref{NWkwPRG-1cMqeO-1}}\RA{}
  protected \LA{}GreedyInsertion: end(0)~{\nwtagstyle{}\subpageref{NWkwPRG-222iyz-1}}\RA{}
  private \LA{}GreedyInsertion: computeCost(6)~{\nwtagstyle{}\subpageref{NWkwPRG-23iwWy-1}}\RA{}
\}
\nwnotused{GreedyInsertion.java}\nwendcode{}\nwbegindocs{2}\nwdocspar

\section{Preamble}
\nwenddocs{}\nwbegincode{3}\sublabel{NWkwPRG-1neuor-1}\nwmargintag{{\nwtagstyle{}\subpageref{NWkwPRG-1neuor-1}}}\moddef{GreedyInsertion.java preamble~{\nwtagstyle{}\subpageref{NWkwPRG-1neuor-1}}}\endmoddef\nwstartdeflinemarkup\nwusesondefline{\\{NWkwPRG-1Tuf6f-1}}\nwenddeflinemarkup
package com.github.jargors.client;
import com.github.jargors.sim.*;
import java.sql.SQLException;
\nwused{\\{NWkwPRG-1Tuf6f-1}}\nwendcode{}\nwbegindocs{4}\nwdocspar

\section{Member Variables}
\nwenddocs{}\nwbegincode{5}\sublabel{NWkwPRG-20xkUX-1}\nwmargintag{{\nwtagstyle{}\subpageref{NWkwPRG-20xkUX-1}}}\moddef{GreedyInsertion: member variables~{\nwtagstyle{}\subpageref{NWkwPRG-20xkUX-1}}}\endmoddef\nwstartdeflinemarkup\nwusesondefline{\\{NWkwPRG-1Tuf6f-1}}\nwenddeflinemarkup
final int PICKUP_THRESHOLD = 600;  // meters
final int MAX_SCHEDULE_LENGTH = 8;
int[] locations = new int[] \{ \};
int count_rejections = 0;
\nwused{\\{NWkwPRG-1Tuf6f-1}}\nwendcode{}\nwbegindocs{6}\nwdocspar

\section{Chunks}

\subsection{Initialize global vars}
\nwenddocs{}\nwbegincode{7}\sublabel{NWkwPRG-jOToH-1}\nwmargintag{{\nwtagstyle{}\subpageref{NWkwPRG-jOToH-1}}}\moddef{Initialize global vars~{\nwtagstyle{}\subpageref{NWkwPRG-jOToH-1}}}\endmoddef\nwstartdeflinemarkup\nwusesondefline{\\{NWkwPRG-2OWvsM-1}}\nwenddeflinemarkup
if (DEBUG) \{
  tools.Print("Extract request \{ rid="+r[0]+", rq="+r[1]+", ro="+r[4]+", rd="+r[5]+" \}");
\}
int     opt_k       = -1;
int     opt_c       = Integer.MAX_VALUE;
int[]   opt_b       = new int[] \{ \};
int[][] opt_cache_t = new int[][] \{ \};
int[][] opt_cache_v = new int[][] \{ \};
if (DEBUG) \{
  tools.Print("Reset opt_k=-1, opt_c=Integer.MAX_VALUE, opt_b=\{\}, opt_cache_t=\{\}, opt_cache_v=\{\}");
\}
final int   T = communicator.retrieveClock();
final int[] L = locations.clone();
final int[] C = tools.filterByHaversine(r[4], L, PICKUP_THRESHOLD);
if (DEBUG) \{
  tools.Print("Initialize T="+T+"; C.length="+C.length);
\}
\nwused{\\{NWkwPRG-2OWvsM-1}}\nwendcode{}\nwbegindocs{8}\nwdocspar

\subsection{Initialize server vars}
\nwenddocs{}\nwbegincode{9}\sublabel{NWkwPRG-3y7kuU-1}\nwmargintag{{\nwtagstyle{}\subpageref{NWkwPRG-3y7kuU-1}}}\moddef{Initialize server vars~{\nwtagstyle{}\subpageref{NWkwPRG-3y7kuU-1}}}\endmoddef\nwstartdeflinemarkup\nwusesondefline{\\{NWkwPRG-2OWvsM-1}}\nwenddeflinemarkup
final int sid = L[(k_cand + 0)];
final int st  = L[(k_cand + 1)];
final int sv  = L[(k_cand + 2)];
if (DEBUG) \{
  tools.Print("Extract server \{ sid="+sid+", st="+st+", sv="+sv+" \}");
\}
int     s_k       = -1;
int     s_c       = Integer.MAX_VALUE;
int[]   s_b       = new int[] \{ \};
int[][] s_cache_t = new int[][] \{ \};
int[][] s_cache_v = new int[][] \{ \};
if (DEBUG) \{
  tools.Print("Reset s_k=-1, s_c=Integer.MAX_VALUE, s_b=\{\}, s_cache_t=\{\}, s_cache_v=\{\}");
\}
final int[] y = communicator.queryServerScheduleRemaining(sid, T);
final int   n = (y.length/4);
final int   z = communicator.queryServerDistanceRemaining(sid, T)[0];
if (DEBUG) \{
  tools.Print("Initialize y.length="+y.length+"; n="+n+"; z="+z);
\}
\nwused{\\{NWkwPRG-2OWvsM-1}}\nwendcode{}\nwbegindocs{10}\nwdocspar

\subsection{Case 1: server is idle or heading towards own destination}
\nwenddocs{}\nwbegincode{11}\sublabel{NWkwPRG-4dVkBS-1}\nwmargintag{{\nwtagstyle{}\subpageref{NWkwPRG-4dVkBS-1}}}\moddef{Case 1: server is idle or heading towards own destination~{\nwtagstyle{}\subpageref{NWkwPRG-4dVkBS-1}}}\endmoddef\nwstartdeflinemarkup\nwusesondefline{\\{NWkwPRG-2OWvsM-1}}\nwenddeflinemarkup
if (n == 1) \{
  if (DEBUG) \{
    tools.Print("Detected n=1");
  \}
  // Beware! Important note! Jargo cannot handle the case where a customer
  // appears at the same vertex that a server is idling at! The reason is
  // because the route would be recorded as
  //   (t1, v)
  //   (t2, v)
  // due to preserving history, (t1, v) cannot be changed; the customer appears
  // at t2; a new waypoint (t2, v) must be recorded in the route in order to be
  // referenceable by Table PD, causing the self-referencing edge (v, v)! This
  // edge violates a Table E constraint.
  //
  // As a workaround, we skip the server if it is idling and happens
  // to be at the request origin.
  if (sv == r[4]) \{
    continue;
  \}

  int[] b = new int[12];
  b[0] = st;
  b[1] = sv;
  b[2] = sid;
  b[4] = r[4];
  b[5] = r[0];
  b[7] = r[5];
  b[8] = r[0];
  b[10] = y[1];
  b[11] = sid;
  if (DEBUG) \{
    tools.Print("Set b[0]="+b[0]);
    tools.Print("Set b[1]="+b[1]);
    tools.Print("Set b[2]="+b[2]);
    tools.Print("Set b[4]="+b[4]);
    tools.Print("Set b[5]="+b[5]);
    tools.Print("Set b[7]="+b[7]);
    tools.Print("Set b[8]="+b[8]);
    tools.Print("Set b[10]="+b[10]);
    tools.Print("Set b[11]="+b[11]);
  \}
  int[][] cache_t = new int[3][];
  int[][] cache_v = new int[3][];
  int c = computeCost(b, cache_t, cache_v, s_c, z, r[2]);
  if (DEBUG) \{
    tools.Print("computeCost returned c="+c);
  \}
  if (c != -1) \{
    \LA{}Accept single solution~{\nwtagstyle{}\subpageref{NWkwPRG-47ILqi-1}}\RA{}
  \}
\}
\nwused{\\{NWkwPRG-2OWvsM-1}}\nwendcode{}\nwbegindocs{12}\nwdocspar

\subsection{Case 2: server is heading towards a customer}
\nwenddocs{}\nwbegincode{13}\sublabel{NWkwPRG-22IadP-1}\nwmargintag{{\nwtagstyle{}\subpageref{NWkwPRG-22IadP-1}}}\moddef{Case 2: server is heading towards a customer~{\nwtagstyle{}\subpageref{NWkwPRG-22IadP-1}}}\endmoddef\nwstartdeflinemarkup\nwusesondefline{\\{NWkwPRG-2OWvsM-1}}\nwenddeflinemarkup
if (n > 1 && n < MAX_SCHEDULE_LENGTH) \{
  if (DEBUG) \{
    tools.Print("Detected n > 1");
  \}
  // Beware! Important note! Jargo cannot handle the case where a customer
  // appears at the same vertex that a server was last seen! In other words,
  // ro cannot equal sv! The reason is that during schedule update, we
  // delete from CQ starting from st onward before re-inserting the new schedule.
  // If there are existing labels on st in CQ, then we would be inserting a
  // second entry on st, causing a constraint violation. If we delete the
  // existing labels, we would have to first query for them and then re-add them.
  // This could be a future fix.
  //
  // As a workaround, we skip the server if it's last-seen vertex happens
  // to be at the request origin.
  if (sv == r[4]) \{
    continue;
  \}

  for (int i = 0; i < n; i++) \{
    for (int j = i; j < n; j++) \{
      boolean capacity_ok = true;
      if (DEBUG) \{
        tools.Print("Set i="+i+", j="+j+", capacity_ok=true");
      \}
      \LA{}Compute sequence~{\nwtagstyle{}\subpageref{NWkwPRG-3KjiZY-1}}\RA{}
      if (capacity_ok) \{
        int[][] cache_t = new int[(b.length/3 - 1)][];
        int[][] cache_v = new int[(b.length/3 - 1)][];
        int c = computeCost(b, cache_t, cache_v, s_c, z, 0);
        if (DEBUG) \{
          tools.Print("computeCost returned c="+c);
        \}
        if (c != -1) \{
          \LA{}Accept single solution~{\nwtagstyle{}\subpageref{NWkwPRG-47ILqi-1}}\RA{}
        \}
      \}
    \}
  \}
\}
\nwused{\\{NWkwPRG-2OWvsM-1}}\nwendcode{}\nwbegindocs{14}\nwdocspar

\subsection{Compute sequence}
\nwenddocs{}\nwbegincode{15}\sublabel{NWkwPRG-3KjiZY-1}\nwmargintag{{\nwtagstyle{}\subpageref{NWkwPRG-3KjiZY-1}}}\moddef{Compute sequence~{\nwtagstyle{}\subpageref{NWkwPRG-3KjiZY-1}}}\endmoddef\nwstartdeflinemarkup\nwusesondefline{\\{NWkwPRG-22IadP-1}}\nwenddeflinemarkup
int[] b = new int[3*(n + 3)];
for (int p = 0; p < (b.length/3); p++) \{
  if (DEBUG) \{
    tools.Print("Set p="+p);
  \}
  \LA{}Check capacity~{\nwtagstyle{}\subpageref{NWkwPRG-27P27q-1}}\RA{}
  if (p == (n + 2)) \{
    b[(3*p + 1)] = y[(4*(p - 3) + 1)];
    b[(3*p + 2)] = y[(4*(p - 3) + 2)];  // server label
  \} else if (p > (j + 2)) \{
    b[(3*p + 1)] = y[(4*(p - 3) + 1)];
    b[(3*p + 2)] = y[(4*(p - 3) + 3)];
  \} else if (p == (j + 2)) \{
    b[(3*p + 1)] = r[5];
    b[(3*p + 2)] = r[0];
  \} else if (p > (i + 1)) \{
    b[(3*p + 1)] = y[(4*(p - 2) + 1)];
    b[(3*p + 2)] = y[(4*(p - 2) + 3)];
  \} else if (p == (i + 1)) \{
    b[(3*p + 1)] = r[4];
    b[(3*p + 2)] = r[0];
  \} else if (p > 0) \{
    b[(3*p + 0)] = y[(4*(p - 1) + 0)];
    b[(3*p + 1)] = y[(4*(p - 1) + 1)];
    b[(3*p + 2)] = y[(4*(p - 1) + 3)];
    if (DEBUG) \{
      tools.Print("Set b["+(3*p + 0)+"]="+b[(3*p + 0)]);
    \}
  \} else \{
    b[(3*p + 0)] = st;
    b[(3*p + 1)] = sv;
    b[(3*p + 2)] = sid;
    if (DEBUG) \{
      tools.Print("Set b["+(3*p + 0)+"]="+b[(3*p + 0)]);
    \}
  \}
  if (DEBUG) \{
    tools.Print("Set b["+(3*p + 1)+"]="+b[(3*p + 1)]);
    tools.Print("Set b["+(3*p + 2)+"]="+b[(3*p + 2)]);
  \}
\}
\nwused{\\{NWkwPRG-22IadP-1}}\nwendcode{}\nwbegindocs{16}\nwdocspar

\subsection{Check capacity}
\nwenddocs{}\nwbegincode{17}\sublabel{NWkwPRG-27P27q-1}\nwmargintag{{\nwtagstyle{}\subpageref{NWkwPRG-27P27q-1}}}\moddef{Check capacity~{\nwtagstyle{}\subpageref{NWkwPRG-27P27q-1}}}\endmoddef\nwstartdeflinemarkup\nwusesondefline{\\{NWkwPRG-3KjiZY-1}}\nwenddeflinemarkup
if (DEBUG) \{
  tools.Print("Check capacity");
\}
if (p >= i && p <= j) \{
  if (DEBUG) \{
    tools.Print("Detected p >= "+i+" && p <= "+j+"; Check capacity");
  \}
  if (r[1] + communicator.queryServerLoadMax(sid, (p == 0 ? st : y[(4*(p - 1))]))[0] > 0) \{
    if (DEBUG) \{
      tools.Print("Detected capacity violation; Break");
    \}
    capacity_ok = false;
    break;
  \}
\}
\nwused{\\{NWkwPRG-3KjiZY-1}}\nwendcode{}\nwbegindocs{18}\nwdocspar

\subsection{Accept single solution}
\nwenddocs{}\nwbegincode{19}\sublabel{NWkwPRG-47ILqi-1}\nwmargintag{{\nwtagstyle{}\subpageref{NWkwPRG-47ILqi-1}}}\moddef{Accept single solution~{\nwtagstyle{}\subpageref{NWkwPRG-47ILqi-1}}}\endmoddef\nwstartdeflinemarkup\nwusesondefline{\\{NWkwPRG-4dVkBS-1}\\{NWkwPRG-22IadP-1}}\nwenddeflinemarkup
if ((c - z) < s_c) \{
  if (DEBUG) \{
    tools.Print("Detected ("+c+" - "+z+")="+(c - z)+" is less than "+s_c);
  \}
  s_k = k_cand;
  s_c = (c - z);
  s_b = b;
  s_cache_t = cache_t;
  s_cache_v = cache_v;
  if (DEBUG) \{
    tools.Print("Replace incumbent single solution, set s_k="+s_k+"; s_c="+s_c+"; s_b.length="+s_b.length);
  \}
  if (s_c < 0) \{
    throw new ClientException("Negative detour");
  \}
\} else \{
  if (DEBUG) \{
    tools.Print("Detected ("+c+" - "+z+")="+(c - z)+" is greater than "+s_c+"; Keep incumbent single solution");
  \}
\}
\nwused{\\{NWkwPRG-4dVkBS-1}\\{NWkwPRG-22IadP-1}}\nwendcode{}\nwbegindocs{20}\nwdocspar

\subsection{Accept server solution}
\nwenddocs{}\nwbegincode{21}\sublabel{NWkwPRG-1EyDWB-1}\nwmargintag{{\nwtagstyle{}\subpageref{NWkwPRG-1EyDWB-1}}}\moddef{Accept server solution~{\nwtagstyle{}\subpageref{NWkwPRG-1EyDWB-1}}}\endmoddef\nwstartdeflinemarkup\nwusesondefline{\\{NWkwPRG-2OWvsM-1}}\nwenddeflinemarkup
if (s_c < opt_c) \{
  if (DEBUG) \{
    tools.Print("Detected "+s_c+" < "+opt_c);
  \}
  opt_k = s_k;
  opt_c = s_c;
  opt_b = s_b;
  opt_cache_t = s_cache_t;
  opt_cache_v = s_cache_v;
  if (DEBUG) \{
    tools.Print("Replace incumbent server solution, set opt_k="+opt_k+"; opt_c="+opt_c+"; opt_b.length="+s_b.length);
  \}
\} else \{
  if (DEBUG) \{
    tools.Print("Detected "+s_c+" > "+opt_c+"; Keep incumbent server solution");
  \}
\}
\nwused{\\{NWkwPRG-2OWvsM-1}}\nwendcode{}\nwbegindocs{22}\nwdocspar

\subsection{Submit global solution}
\nwenddocs{}\nwbegincode{23}\sublabel{NWkwPRG-29TDYN-1}\nwmargintag{{\nwtagstyle{}\subpageref{NWkwPRG-29TDYN-1}}}\moddef{Submit global solution~{\nwtagstyle{}\subpageref{NWkwPRG-29TDYN-1}}}\endmoddef\nwstartdeflinemarkup\nwusesondefline{\\{NWkwPRG-2OWvsM-1}}\nwenddeflinemarkup
if (opt_k != -1) \{
  \LA{}Construct w from cache~{\nwtagstyle{}\subpageref{NWkwPRG-4ACdbX-1}}\RA{}
  int sid = L[opt_k];
  int[] rids = new int[] \{ r[0] \};
  if (DEBUG) \{
    tools.Print("Submit sid="+sid+"; opt_w.length="+opt_w.length
      +"; opt_b.length="+opt_b.length+"; opt_c="+opt_c+"; rids.length="+rids.length);
  \}
  try \{
    // We added the server's current location to b to help us compute cost,
    // but now we remove the location because it's not really a part of the
    // schedule.
    int[] opt_opt_b = new int[(opt_b.length - 3)];
    for (int i = 3; i < opt_b.length; i++) \{
      opt_opt_b[(i - 3)] = opt_b[i];
    \}
    communicator.updateServerService(sid, opt_w, opt_opt_b, rids, new int[] \{ \});
  \} catch (RouteIllegalOverwriteException e) \{
    count_rejections++;
    if (DEBUG) \{
      tools.Print("Submission rejected due to illegal overwrite!");
    \}
  \} catch (TimeWindowException e) \{
    count_rejections++;
    if (DEBUG) \{
      tools.Print("Submission rejected due to time window violation");
      tools.Print(e.toString());
    \}
  \}
\} else \{
  if (DEBUG) \{
    tools.Print("No match found");
  \}
  communicator.forwardReturnRequest(r);
\}
\nwused{\\{NWkwPRG-2OWvsM-1}}\nwendcode{}\nwbegindocs{24}\nwdocspar

\subsection{Construct w from cache}
\nwenddocs{}\nwbegincode{25}\sublabel{NWkwPRG-4ACdbX-1}\nwmargintag{{\nwtagstyle{}\subpageref{NWkwPRG-4ACdbX-1}}}\moddef{Construct w from cache~{\nwtagstyle{}\subpageref{NWkwPRG-4ACdbX-1}}}\endmoddef\nwstartdeflinemarkup\nwusesondefline{\\{NWkwPRG-29TDYN-1}}\nwenddeflinemarkup
int w_len = 1;
for (int[] leg : opt_cache_v) \{
  w_len += (leg.length - 1);
\}
w_len *= 2;
if (DEBUG) \{
  tools.Print("Construct w from cache");
  tools.Print("Initialize w.length="+w_len);
\}
int[] opt_w = new int[w_len];
opt_w[0] = opt_cache_t[0][0];
opt_w[1] = opt_cache_v[0][0];
if (DEBUG) \{
  tools.Print("Set opt_w[0]="+opt_w[0]);
  tools.Print("Set opt_w[1]="+opt_w[1]);
\}
int base = 0;
for (int p = 0; p < opt_cache_v.length; p++) \{
  if (p > 0) \{
    base += 2*(opt_cache_v[(p - 1)].length - 1);
  \}
  for (int q = 1; q < opt_cache_v[p].length; q++) \{
    opt_w[(base + 2*q + 0)] = opt_cache_t[p][q];
    opt_w[(base + 2*q + 1)] = opt_cache_v[p][q];
    if (DEBUG) \{
      tools.Print("Set opt_w["+(base + 2*q + 0)+"]="+opt_w[(base + 2*q + 0)]);
      tools.Print("Set opt_w["+(base + 2*q + 1)+"]="+opt_w[(base + 2*q + 1)]);
    \}
  \}
\}
\nwused{\\{NWkwPRG-29TDYN-1}}\nwendcode{}\nwbegindocs{26}\nwdocspar

\subsection{Check feasible cost}
\nwenddocs{}\nwbegincode{27}\sublabel{NWkwPRG-UGv0L-1}\nwmargintag{{\nwtagstyle{}\subpageref{NWkwPRG-UGv0L-1}}}\moddef{Check feasible cost~{\nwtagstyle{}\subpageref{NWkwPRG-UGv0L-1}}}\endmoddef\nwstartdeflinemarkup\nwusesondefline{\\{NWkwPRG-23iwWy-1}}\nwenddeflinemarkup
if ((c - z) > s_c) \{
  if (DEBUG) \{
    tools.Print("Detected cost infeasible ("+(c - z)+" > "+s_c+"); Return");
  \}
  return -1;
\}
\nwused{\\{NWkwPRG-23iwWy-1}}\nwendcode{}\nwbegindocs{28}\nwdocspar

\subsection{Check feasible time window}
\nwenddocs{}\nwbegincode{29}\sublabel{NWkwPRG-4PssH0-1}\nwmargintag{{\nwtagstyle{}\subpageref{NWkwPRG-4PssH0-1}}}\moddef{Check feasible time window~{\nwtagstyle{}\subpageref{NWkwPRG-4PssH0-1}}}\endmoddef\nwstartdeflinemarkup\nwusesondefline{\\{NWkwPRG-23iwWy-1}}\nwenddeflinemarkup
int[] u = communicator.queryUser(b[(3*p + 2)]);
if (d < u[2] || d > u[3]) \{
  if (DEBUG) \{
    tools.Print("Detected time infeasible for user "+b[(3*p + 2)]+"; d="+d+"; u[2]="+u[2]+"; u[3]="+u[3]+"; Return");
  \}
  return -1;
\}
\nwused{\\{NWkwPRG-23iwWy-1}}\nwendcode{}\nwbegindocs{30}\nwdocspar


\section{Methods}

\subsection{GreedyInsertion: \texttt{handleRequest}(1)}
\nwenddocs{}\nwbegincode{31}\sublabel{NWkwPRG-2OWvsM-1}\nwmargintag{{\nwtagstyle{}\subpageref{NWkwPRG-2OWvsM-1}}}\moddef{GreedyInsertion: handleRequest(1)~{\nwtagstyle{}\subpageref{NWkwPRG-2OWvsM-1}}}\endmoddef\nwstartdeflinemarkup\nwusesondefline{\\{NWkwPRG-1Tuf6f-1}}\nwenddeflinemarkup
void handleRequest(int[] r) throws ClientException, ClientFatalException \{
  try \{
    \LA{}Initialize global vars~{\nwtagstyle{}\subpageref{NWkwPRG-jOToH-1}}\RA{}
    for (int k_cand : C) \{
      \LA{}Initialize server vars~{\nwtagstyle{}\subpageref{NWkwPRG-3y7kuU-1}}\RA{}
      \LA{}Case 1: server is idle or heading towards own destination~{\nwtagstyle{}\subpageref{NWkwPRG-4dVkBS-1}}\RA{}
      \LA{}Case 2: server is heading towards a customer~{\nwtagstyle{}\subpageref{NWkwPRG-22IadP-1}}\RA{}
      if (n >= MAX_SCHEDULE_LENGTH) \{
        if (DEBUG) \{
          tools.Print("Detected n >= MAX_SCHEDULE_LENGTH");
        \}
      \}
      \LA{}Accept server solution~{\nwtagstyle{}\subpageref{NWkwPRG-1EyDWB-1}}\RA{}
    \}
    \LA{}Submit global solution~{\nwtagstyle{}\subpageref{NWkwPRG-29TDYN-1}}\RA{}
  \} catch (SQLException e) \{
    throw new ClientException(e);
  \} catch (VertexNotFoundException | EdgeNotFoundException | UserNotFoundException e) \{
    throw new ClientException(e);
  \}
\}
\nwused{\\{NWkwPRG-1Tuf6f-1}}\nwendcode{}\nwbegindocs{32}\nwdocspar

\subsection{GreedyInsertion: \texttt{endCollectServerLocations}(1)}
\nwenddocs{}\nwbegincode{33}\sublabel{NWkwPRG-1cMqeO-1}\nwmargintag{{\nwtagstyle{}\subpageref{NWkwPRG-1cMqeO-1}}}\moddef{GreedyInsertion: endCollectServerLocations(1)~{\nwtagstyle{}\subpageref{NWkwPRG-1cMqeO-1}}}\endmoddef\nwstartdeflinemarkup\nwusesondefline{\\{NWkwPRG-1Tuf6f-1}}\nwenddeflinemarkup
void endCollectServerLocations(int[] src) \{
  locations = src.clone();
\}
\nwused{\\{NWkwPRG-1Tuf6f-1}}\nwendcode{}\nwbegindocs{34}\nwdocspar

\subsection{GreedyInsertion:\texttt{end}(0)}
\nwenddocs{}\nwbegincode{35}\sublabel{NWkwPRG-222iyz-1}\nwmargintag{{\nwtagstyle{}\subpageref{NWkwPRG-222iyz-1}}}\moddef{GreedyInsertion: end(0)~{\nwtagstyle{}\subpageref{NWkwPRG-222iyz-1}}}\endmoddef\nwstartdeflinemarkup\nwusesondefline{\\{NWkwPRG-1Tuf6f-1}}\nwenddeflinemarkup
void end() \{
  tools.Print("Count rejections: "+count_rejections);
\}
\nwused{\\{NWkwPRG-1Tuf6f-1}}\nwendcode{}\nwbegindocs{36}\nwdocspar

\subsection{GreedyInsertion: \texttt{computeCost}(6)}
\nwenddocs{}\nwbegincode{37}\sublabel{NWkwPRG-23iwWy-1}\nwmargintag{{\nwtagstyle{}\subpageref{NWkwPRG-23iwWy-1}}}\moddef{GreedyInsertion: computeCost(6)~{\nwtagstyle{}\subpageref{NWkwPRG-23iwWy-1}}}\endmoddef\nwstartdeflinemarkup\nwusesondefline{\\{NWkwPRG-1Tuf6f-1}}\nwenddeflinemarkup
int computeCost(int[] b, int[][] cache_t, int[][] cache_v, int s_c, int z, int d_init)
throws ClientException, ClientFatalException \{
  try \{
    if (DEBUG) \{
      tools.Print("computeCost called on b.length="+b.length+", s_c="+s_c+", z="+z);
    \}
    int c = 0;
    int d = 0;
    int[] leg = new int[] \{ \};
    int[] ddnu = new int[] \{ \};
    if (DEBUG) \{
      tools.Print("..set c=0");
    \}
    for (int p = 1; p < (b.length/3); p++) \{
      if (DEBUG) \{
        tools.Print("..set p="+p);
      \}
      leg = tools.computeShortestPath(b[(3*(p - 1) + 1)], b[(3*p + 1)]);
      if (DEBUG) \{
        tools.Print("..call computeShortestPath("+b[(3*(p - 1) + 1)]+", "+b[(3*p + 1)]+")");
      \}
      d = b[(3*(p - 1))];
      if (DEBUG) \{
        tools.Print("..set d="+d);
      \}
      cache_t[(p - 1)] = new int[leg.length];
      cache_t[(p - 1)][0] = d;
      if (DEBUG) \{
        tools.Print("..set cache_t["+(p - 1)+"]=new int["+leg.length+"]");
        tools.Print("..set cache_t["+(p - 1)+"][0]="+d);
      \}
      if (p == 1) \{
        d += d_init;
        if (DEBUG) \{
          tools.Print("..add d_init; d="+d);
        \}
      \}
      for (int q = 1; q < leg.length; q++) \{
        if (DEBUG) \{
          tools.Print("....set q="+q);
        \}
        ddnu = communicator.queryEdge(leg[(q - 1)], leg[q]);
        if (DEBUG) \{
          tools.Print("....call queryEdge("+leg[(q - 1)]+", "+leg[q]+")");
        \}
        c += ddnu[0];
        d += tools.computeDuration(ddnu[0], ddnu[1]);
        if (DEBUG) \{
          tools.Print("....set c="+c);
          tools.Print("....set d="+d);
        \}
        cache_t[(p - 1)][q] = d;
        if (DEBUG) \{
          tools.Print("....set cache_t["+(p - 1)+"]["+q+"]="+d);
        \}
      \}
      \LA{}Check feasible cost~{\nwtagstyle{}\subpageref{NWkwPRG-UGv0L-1}}\RA{}
      \LA{}Check feasible time window~{\nwtagstyle{}\subpageref{NWkwPRG-4PssH0-1}}\RA{}
      b[(3*p)] = d;
      if (DEBUG) \{
        tools.Print("..set b["+(3*p)+"]="+d);
      \}
      cache_v[(p - 1)] = leg.clone();
      if (DEBUG) \{
        tools.Print("..set cache_v["+(p - 1)+"]=<leg.length="+leg.length+">");
      \}
    \}
    return c;
  \} catch (SQLException e) \{
    tools.Print("Something very bad happened");
    tools.PrintSQLException(e);
    throw new ClientFatalException();
  \} catch (GtreeNotLoadedException e) \{
    tools.Print("Gtree not loaded, can't continue!");
    throw new ClientFatalException();
  \} catch (GtreeIllegalSourceException e) \{
    throw new ClientException(e);
  \} catch (EdgeNotFoundException | UserNotFoundException e) \{
    throw new ClientException(e);
  \}
\}
\nwused{\\{NWkwPRG-1Tuf6f-1}}\nwendcode{}\nwbegindocs{38}\nwdocspar

\nwenddocs{}\nwfilename{src/traffic-overview.nw}\nwbegindocs{0}\part{Traffic Functions}
\label{part-traffic}

\chapter{Overview}
\label{traffic-overview}

\nwenddocs{}\nwfilename{src/traffic-broadway.nw}\nwbegindocs{0}\chapter{Traffic: Broadway}

\nwenddocs{}\nwbegincode{1}\sublabel{NW1QgbvE-3asg0V-1}\nwmargintag{{\nwtagstyle{}\subpageref{NW1QgbvE-3asg0V-1}}}\moddef{Broadway.java~{\nwtagstyle{}\subpageref{NW1QgbvE-3asg0V-1}}}\endmoddef\nwstartdeflinemarkup\nwenddeflinemarkup
package com.github.jargors.traffic;
import com.github.jargors.sim.*;
import java.util.Map;
public class Broadway extends Traffic \{
  private Map<Integer, Integer> lu_broadway = Map.ofEntries(
    Map.entry(10024,6898),
    Map.entry(10071,7512),
    Map.entry(10109,11306),
    Map.entry(10298,4057),
    Map.entry(10342,9159),
    Map.entry(10378,10775),
    Map.entry(1041,1043),
    Map.entry(1043,5145),
    Map.entry(10506,11594),
    Map.entry(10517,10519),
    Map.entry(10519,3814),
    Map.entry(10565,9347),
    Map.entry(10572,3304),
    Map.entry(10628,6944),
    Map.entry(10671,10672),
    Map.entry(10672,326),
    Map.entry(10690,10691),
    Map.entry(10691,1994),
    Map.entry(10729,11922),
    Map.entry(10775,4882),
    Map.entry(10864,2072),
    Map.entry(10949,7023),
    Map.entry(10966,7019),
    Map.entry(10983,2568),
    Map.entry(11024,3875),
    Map.entry(11066,7349),
    Map.entry(11172,8494),
    Map.entry(11197,1893),
    Map.entry(11306,7781),
    Map.entry(11314,2490),
    Map.entry(11366,1216),
    Map.entry(11398,10342),
    Map.entry(11438,8502),
    Map.entry(11439,11398),
    Map.entry(1155,6168),
    Map.entry(11594,160),
    Map.entry(1162,2350),
    Map.entry(11922,10949),
    Map.entry(11930,7340),
    Map.entry(12,4286),
    Map.entry(12114,1239),
    Map.entry(1216,1218),
    Map.entry(1218,7106),
    Map.entry(1239,1241),
    Map.entry(1241,3383),
    Map.entry(1290,4347),
    Map.entry(1467,1469),
    Map.entry(1469,10628),
    Map.entry(1564,1565),
    Map.entry(1565,4163),
    Map.entry(160,161),
    Map.entry(161,8444),
    Map.entry(1732,1734),
    Map.entry(1734,1564),
    Map.entry(180,181),
    Map.entry(181,4284),
    Map.entry(1821,10671),
    Map.entry(1832,1833),
    Map.entry(1833,10864),
    Map.entry(1854,1855),
    Map.entry(1855,2920),
    Map.entry(1883,6805),
    Map.entry(1893,1894),
    Map.entry(1894,3858),
    Map.entry(1936,1937),
    Map.entry(1937,5830),
    Map.entry(1950,887),
    Map.entry(1994,632),
    Map.entry(2072,2073),
    Map.entry(2073,7294),
    Map.entry(2083,2084),
    Map.entry(2084,6989),
    Map.entry(2118,7558),
    Map.entry(2146,2147),
    Map.entry(2147,3515),
    Map.entry(218,219),
    Map.entry(219,11314),
    Map.entry(2194,2195),
    Map.entry(2195,8923),
    Map.entry(2252,4641),
    Map.entry(226,8293),
    Map.entry(2277,2278),
    Map.entry(2278,9841),
    Map.entry(2316,843),
    Map.entry(2323,2324),
    Map.entry(2324,8683),
    Map.entry(2350,12),
    Map.entry(2393,2394),
    Map.entry(2394,4693),
    Map.entry(2413,2083),
    Map.entry(2417,6),
    Map.entry(2418,2419),
    Map.entry(2419,6973),
    Map.entry(2458,1854),
    Map.entry(2468,2418),
    Map.entry(2490,2492),
    Map.entry(2492,4298),
    Map.entry(2568,2570),
    Map.entry(2570,10024),
    Map.entry(2615,10729),
    Map.entry(2680,1883),
    Map.entry(2750,2751),
    Map.entry(2751,3016),
    Map.entry(2888,12114),
    Map.entry(2920,2888),
    Map.entry(2928,2194),
    Map.entry(3011,3013),
    Map.entry(3013,2417),
    Map.entry(3016,4767),
    Map.entry(3024,9569),
    Map.entry(3107,1467),
    Map.entry(3205,3207),
    Map.entry(3207,9193),
    Map.entry(3213,582),
    Map.entry(323,325),
    Map.entry(325,10109),
    Map.entry(326,327),
    Map.entry(327,3652),
    Map.entry(3299,2468),
    Map.entry(3304,3305),
    Map.entry(3305,5836),
    Map.entry(3383,6221),
    Map.entry(3423,3424),
    Map.entry(3424,6985),
    Map.entry(3432,3434),
    Map.entry(3434,323),
    Map.entry(3471,1936),
    Map.entry(351,353),
    Map.entry(3515,2615),
    Map.entry(353,773),
    Map.entry(355,356),
    Map.entry(3559,7418),
    Map.entry(356,11366),
    Map.entry(3597,3598),
    Map.entry(3598,8635),
    Map.entry(3628,3629),
    Map.entry(3629,3107),
    Map.entry(363,364),
    Map.entry(364,6983),
    Map.entry(3652,9572),
    Map.entry(3670,8411),
    Map.entry(3814,3816),
    Map.entry(3816,7084),
    Map.entry(3858,6061),
    Map.entry(3875,3876),
    Map.entry(3876,2458),
    Map.entry(398,399),
    Map.entry(399,3471),
    Map.entry(4057,2277),
    Map.entry(4149,2316),
    Map.entry(4163,6214),
    Map.entry(4201,4203),
    Map.entry(4203,5670),
    Map.entry(4213,6243),
    Map.entry(4284,5690),
    Map.entry(4286,6625),
    Map.entry(4298,9435),
    Map.entry(4347,4348),
    Map.entry(4348,6066),
    Map.entry(4357,4358),
    Map.entry(4358,6855),
    Map.entry(4414,8229),
    Map.entry(4516,4620),
    Map.entry(4529,1155),
    Map.entry(4535,5625),
    Map.entry(4543,8810),
    Map.entry(4620,3597),
    Map.entry(4641,4881),
    Map.entry(4693,4694),
    Map.entry(4694,5408),
    Map.entry(4742,4743),
    Map.entry(4743,3024),
    Map.entry(4767,4768),
    Map.entry(4768,5748),
    Map.entry(4774,4775),
    Map.entry(4775,5797),
    Map.entry(4881,7348),
    Map.entry(4882,4883),
    Map.entry(4883,4357),
    Map.entry(4927,11930),
    Map.entry(4932,5186),
    Map.entry(5017,7725),
    Map.entry(5024,5861),
    Map.entry(5145,5146),
    Map.entry(5146,525),
    Map.entry(5183,5184),
    Map.entry(5184,218),
    Map.entry(5186,4543),
    Map.entry(525,526),
    Map.entry(526,2118),
    Map.entry(5408,351),
    Map.entry(5481,9498),
    Map.entry(5625,4414),
    Map.entry(5670,11197),
    Map.entry(5690,5691),
    Map.entry(5691,10690),
    Map.entry(5797,4149),
    Map.entry(582,583),
    Map.entry(583,8426),
    Map.entry(5830,355),
    Map.entry(5836,11066),
    Map.entry(5861,5862),
    Map.entry(5862,1950),
    Map.entry(6,8),
    Map.entry(605,606),
    Map.entry(606,4529),
    Map.entry(6061,3628),
    Map.entry(6066,10506),
    Map.entry(6076,1162),
    Map.entry(6130,6131),
    Map.entry(6131,8353),
    Map.entry(6168,7647),
    Map.entry(621,11024),
    Map.entry(6214,3423),
    Map.entry(6221,6222),
    Map.entry(6222,7324),
    Map.entry(6234,2393),
    Map.entry(6243,4774),
    Map.entry(632,6750),
    Map.entry(6625,5024),
    Map.entry(6708,8140),
    Map.entry(6750,6752),
    Map.entry(6752,8339),
    Map.entry(6805,3299),
    Map.entry(6855,6856),
    Map.entry(6856,1041),
    Map.entry(6889,4932),
    Map.entry(6944,11438),
    Map.entry(6973,11439),
    Map.entry(6983,6984),
    Map.entry(6984,8154),
    Map.entry(6985,927),
    Map.entry(6989,3011),
    Map.entry(7019,4742),
    Map.entry(7023,5183),
    Map.entry(7084,10983),
    Map.entry(7086,888),
    Map.entry(7106,7645),
    Map.entry(7217,10298),
    Map.entry(7254,2928),
    Map.entry(7294,7295),
    Map.entry(7295,7524),
    Map.entry(7324,3205),
    Map.entry(7328,7330),
    Map.entry(7330,2750),
    //Map.entry(7339,130),
    Map.entry(7339,7328),
    Map.entry(7340,8109),
    Map.entry(7348,10565),
    Map.entry(7349,3559),
    Map.entry(7418,11172),
    Map.entry(7477,10378),
    Map.entry(7512,7513),
    Map.entry(7513,7254),
    Map.entry(7524,605),
    Map.entry(7558,1832),
    Map.entry(7611,9865),
    Map.entry(7645,7646),
    Map.entry(7646,3213),
    Map.entry(7647,4201),
    Map.entry(7725,7775),
    Map.entry(773,2252),
    Map.entry(7775,9024),
    Map.entry(7781,130),
    Map.entry(7853,7855),
    Map.entry(7855,398),
    Map.entry(8,4535),
    Map.entry(8059,10517),
    Map.entry(8109,2919),
    Map.entry(8140,10966),
    Map.entry(8154,6076),
    Map.entry(8201,8202),
    Map.entry(8202,2680),
    Map.entry(8229,8918),
    Map.entry(8293,2413),
    Map.entry(8335,180),
    Map.entry(8339,2323),
    Map.entry(8353,8354),
    Map.entry(8354,7853),
    Map.entry(8411,6234),
    Map.entry(8426,1821),
    Map.entry(843,845),
    Map.entry(8444,363),
    Map.entry(845,3670),
    Map.entry(8475,5017),
    Map.entry(8494,8495),
    Map.entry(8495,6130),
    Map.entry(8502,2146),
    Map.entry(8635,1290),
    Map.entry(8683,8684),
    Map.entry(8684,10572),
    Map.entry(8810,8812),
    Map.entry(8812,9103),
    Map.entry(887,6708),
    Map.entry(888,889),
    Map.entry(889,226),
    Map.entry(8918,7611),
    Map.entry(8923,6889),
    Map.entry(9024,8059),
    Map.entry(9103,4516),
    Map.entry(9159,5481),
    Map.entry(9193,10071),
    Map.entry(927,929),
    Map.entry(929,8475),
    Map.entry(9347,9348),
    Map.entry(9348,8335),
    Map.entry(9435,3432),
    Map.entry(9498,7217),
    Map.entry(9521,621),
    Map.entry(9569,4927),
    Map.entry(9572,9573),
    Map.entry(9573,9908),
    Map.entry(9841,9521),
    Map.entry(9865,9866),
    Map.entry(9866,1732),
    Map.entry(9908,8201)
  );
  // New Year's Eve in Manhattan: speed on Broadway is at 80% at 11:30 PM, then
  // speed decreases starting from midnight until 1:30 AM to 20% until 5:00 AM.
  // We do (vertex - 1) because edges above are 0-indexed while nodes in Jargo
  // are 1-indexed.
  public double apply(int v1, int v2, long msec) \{
    double per = 1.0;
    if ((lu_broadway.containsKey(v1-1) && lu_broadway.get(v1-1) == v2-1)
     || (lu_broadway.containsKey(v2-1) && lu_broadway.get(v2-1) == v1-1)) \{
      if (msec > 84600_000) \{  // 20% at 11:30 PM
        // per = 1-0.8*(86400_000 - msec)/1800_000;
        per = 0.2;
      \} else if (msec < 5400_000) \{  // 100% at 12:00AM moving to 20% at 1:30 AM
        per = Math.max(0.2, (5400_000 - msec)/5400_000);
      \} else if (msec < 18000_000) \{  // 20% from 1:30 AM to 5:00 AM
        per = 0.2;
      \}
    \}
    return per;
  \}
\}
\nwnotused{Broadway.java}\nwendcode{}\nwbegindocs{2}\nwdocspar

\nwenddocs{}
